\chapter{Refugee Camps}
\label{chp:refugeecamps} 
Our initial approach for the report was to look into refugee camps, with the focus on Norwegian relief organizations. The aim was to get an understanding of how refugee camps are run and how the communication is both within the camp and to the outside world. We wanted to look into how the Mesh Potatoes could be utilized to better the communication. We conducted some research trying to get an overview and general understanding. In addition we carried out interviews with people form Norwegian organizations namely Care and the Norwegian Refuge Council. This gave us more information and understanding on the area, but also showed us that refugee camps is an extremely large area, and the differences in the unique camps are huge depending on country and government law, size of the camp, lifetime of the camp, etc. In some countries the Internet and GSM infrastructure is well established, and people have smartphones and laptops. While in other countries this is far from the case, and their only way of receiving news is buy radio or mostly by word of mouth. Some camps are under strong technological restriction because of country law. 

The differences are so big it is hard to find general information, simply because there are no general information. No camp are the same, no situation can be compared to each other. Which made it hard to decide in what direction to focus. A research, as the one first intended, would require a close collaboration with a relief organization to give full proceeds both to us and the research we are conducting on behalf of Village Telco. It was also difficult to create a stable connection and collaboration with one of the Norwegian help organizations. A collaboration like this would require a lot more time than what we have a disposal. 
  
During our research and work in the first months we could see our report had to go in a different direction. The area became to big to get a grip on. We decided to direct out focus onto natural disasters, and the how a emergency box can be used to quickly get internet access and in that way coordinate help, more on this in chapter \ref{chp:quickrollout}.
This chapter contains the research we have done on refugee camps. We will go through some general statistics to get an idea of the life in the camps and the development the over the last years, and summaries of the interviews with Care and Norwegian Refugee Council.

We therefore chose to co-operate with a smaller organization called Making Change. Making Change is a non profit relief organization where military veterans can use their experience to make a change somewhere in the world.

\section{Refugee Camps}
%Generell fakta

Women are in general in a more vulnerable position when living in a  camp, especially if they are single mothers. They may be solely responsible for taking care of the children in addition to sick and elderly family members, maintaining the household, preparing food, acquire water, and securing firewood. Collecting firewood for cooking is a necessity, but it forces women  to walk far away, hence making them vulnerable for sexual assault. they have to turn to sex and other unhealthy and dangerous means in order to survive. \cite{womenRefugee} 


The means of communication vary greatly in the different camps. Some have internet connection and satellite TV, other barely have access to a radio. Even though radios are the most common media for communication, it is not given that all citizens in a camp have access to one. Often people would gather around the few radios that exists in a camp. One issue that limits the usage is the batteries. They are very expensive and hard to acquire. The women were interested in news regarding their place of origin. 

Information walls and word of mouth is often used in order to spread practical information within the camp and about camp activities. Word of mouth is also used in order to retrieve information about the world outside the camp. People visiting the camp were used as sources for information. Earlier studies have shown that social connections with neighbours works as an important medium to transport information, resources and services between individuals. These kind of networking have been used to find lost family members in big camps, as well as get financial help from abroad \cite{womenRefugee}.  



\paragraph{Cell Phone}
The use of cell phones are increasing. Even though the prises are extremely high it does not stop people from calling relatives in Europe and other places in the world. 
\cite{womenRefugee} 


\section{Statistics}
When we talk about refugees it is important to divide between different types of refugees, mainly a refugee and an IDP. 

\paragraph{Refugee.} The definition of a refugee is a person who have been forced to leave their home country because of war, violence or persecution. A refugee often has a justifiable fear of persecution for reasons as religion, political opinion, race, nationality or membership in a certain social group. For these reasons are not able to, or are afraid to  return to their home country. The leading reasons for refugees flee their home country is war and ethnic, religious and tribal violence \cite{refugeeDef}.

\paragraph{Internally Displaced Person.} An internally displaced person (IDP) is a person that has been forced to leave their home and village for some reason and are a refugee in its own country. The man distinction between an IDP and a refugee is that the person has not crossed any country borders. Unlike refugees, the IDPs are not protected by any international laws nor are able to receive many types aids. in the last years the number of IDPs have drastically increased, mostly due to the conflicts between countries. 

A stateless person is someone who does not have citizenship in any country. a citizenship is a legal bond between an individual and the government in that country. 

By the end of 2012 45.2 million people were displaced by force. According to UNHCR is this the largest number in 20 years. The report show that 55 \% of the registered refugees came from countries affected by war as Syria, Afghanistan, Sudan, Iraq and Somalia. The crisis in Syria has been a major factor to displacement, the whole of 647,000 people have be forced out of the country \citep{UNHCRstat}.

In 2012 UNHCR registered 21,300 individual asylum applications from children that either were unaccompanied or separated from their family. 


\section{Interview with CARE - Dadaab Refugee Camp}
We got in contact with Mary Muia from CARE. She is a program assistant at CARE International in Dadaab, Kenya. We sent her a questionnaire with questions about Dadaab refugee camp, with focus on means of communication. The following paragraphs contains information both from different articles referred to in the text, and from the answers from the questionnaire. See Appendix \ref{chp:appendixA} for the full questionnaire. Dadaab is the largest refugee camp in the world, and is located in Daadaab, Kenya \cite{dadaab}. It was created in 1991 \cite{dadaabcare}. Dadaab was created by the government of Kenya and UNHCR to host Somali refugees displaced by civil war. Over the years, the camps have also hosted other nationalities, from the Horn of Africa, the Great Lakes and East African regions. These people constitute less than two percent of the camp population. In April, 2013, there were 423,496 registered refugees in the Dadaab camps. 51 \% of these were female and 58 \% were younger than 18 years old. Also in 2013, UNHCR and its partners decided to conduct a verification exercise to ascertain the current population. The reason for this was that many of those who had arrived in 2011 due to the famine had returned home. As of February, 2014, the current population stands at 369,294. The lead agency for this camp is the UN High Commission for Refugees (UNHCR) \cite{dadaab}. In addition to UNHCR, major international humanitarian agencies like Care, Save the Children and the International Rescue Committee are active helpers in the Dadaab refugee camp. These agencies provide the refugees with critical services (e.g. food, housing, sanitation and medical help). This is an extremely challenging task in refugee camps, especially when they reach this size. During the recent years, the terror group Al Shabaad (Somali-based) have intensified their misdeeds in and around the Dadaab refugee camps. This has made the situation even tougher for the refugees and the relief agencies. 
Muia stated that the biggest challenges in the camps are lack of enough space to accommodate everyone, and lack of enough funds to take care of all the needs of the refugees. Another challenge is the language barrier between the humanitarian staff and the refugees. Many of the staff members neither speak nor understand the Somali language, and as many as 95.6\% of the refugees are Somali. 

Muia explains how the registration process is handled; When a new refugee enters the camp, the refugee reports to a UNHCR reception desk. There the refugee is given a temporary registration, while pending full registration. Upon arrival, the refugees are given information about available services, and which agency is handling what service. Immunizations, medical attention, emergency food supply, tarpaulins, sleeping mats, jerrycans for fetching water and kitchen sets are issued to new arrivals. This is to help them start their new lives in the camp. 

To improve the situation in Dadaab, communication is crucial. In 2011, a group consisting of people from NetHope, Inveneo and the USAID Global Broadband and Innovations Program gathered to discuss ways to improve the means of communication in Dadaab \cite{dadaab}. NetHope is a consortium of over 30 international Non-Governmental Organizations (NGOs) \cite{nethope}. NetHope works with improving connectivity, with the help of information technology, among relief agencies. The aim of this project, called DadaabConnect, was to bring forward more reliable Internet, and find ways for agencies to communicate better internally \cite{dadaab}. The group put together teams that travelled to Kenya to investigate the conditions in the refugee camps, and to find out what they could implement. It was clear from the feedback they got that a better communication system was needed, and that it would make the humanitarian work much easier. It would improve the coordination and the security in the camps. Improvements of these aspects gives the humanitarian agencies better working conditions, and makes it easier for them to help the refugees with critical services. Inveneo started working with Cisco's Tactical Operations (TacOps) to install and configure a local high-speed network \cite{dadaabinveneo}. They also entered a partnership with a local Kenyan mobile and landline telecommunications service provider called Orange. The reason for this was that they wanted to extend the Dadaab compound with new data services. This could be done by using Inveneo's long-distance Wi-Fi solutions. The data services that were added included services requested from the Dadaab aid community. "DadaabNet", a high-speed network, was created in cooperation between Inveneo and TacOps. This network connected the NGOs locally, and made it possible for the agencies to easier communicate internally (VoIP telephony, file sharing etc.). Following this, in March 2012, they started the training of technicians. These technicians were people from Orange, from the technical staff of the NGOs and from Inveneo's staff. The training took place both in classrooms and in the field, in order to give the technicians a wide understanding. The results from DadaabConnect has been great. The humanitarian agencies has gotten better working conditions, due to the improvements in means of communication. Other positive outcomes is that the network is more reliable and cost effective. 

Muia did not specifically mention this project in her answers, but answers on our questions about means of communication within the camp, and with the outside world. She states that CARE as an organization has invested in communication systems in cooperation with ISPs in the capital city of Kenya, Nairobi. Through this cooperation the camp staff are assured to get access to Internet for both official and social use. Several Kenyan telecommunication companies have put up equipment in the camp area, and the camps are therefore provided with access to mobile communication and Internet. Although this is set up, Internet and telephone service outages are fairly common. In addition to mobile communication and Internet, there are radio station services and access to digital television. CARE use telephone services to reach out to refugee staff. 50\% of the refugees have access to mobile phone services. Posters and radio are also used to reach out. Word by mouth (e.g. over speakers) is also a communication technique employed. There are two main telecommunication providers in Dadaad, hence little competition. This makes the prices higher. We asked Muia how the refugees can afford having their own mobile phone, when the costs are high. She says that many refugees have been in the camps for a long time, and therefore have had the time to establish small businesses which gives them some profit. While others get money sent from their relatives. 


\section{Interview with Norwegian Refugee Council}
%hun mener vi skal klassifiere for oppgaven sin del av vi snakker om offisielle leirer. UNHCR har ansvar da. Monitirere enklere da om de tingene er på plass, og evt. hva slags system de bruker. 

We had a Skype interview March 12, 2014, with Katrine Wold from the Norwegian Refugee Council (NRC). The aim of the interview was to hear a little bit about her work in refugee camps and how the situation in the refugee camps are today, with main focus on means of communication.
Katrine Wold has been working for NRC for many years, and also has a background from United Nations (UN). She has worked in emergency and crisis situations abroad. She is specialized in camp management and coordination. In recent years she has been responsible for education, and have had the main focus on youth. We asked her which refugee camps NRC is working in, but she could not give us a clear answer on that question. The reason for this is that NRC works in over 24 countries, and have, as of 2013, reached out to 4.4 million people. She makes it clear that there is a difference between internally displaced persons (IDPs) and refugees. An official refugee must cross a boarder, or else you are internally displaced. NRC works both with refugees and IDPs, and also with people who are affected by having refugees in their local area. NRC does not only help with operational issues in the camps, but they mainly offer services the refugees need. When dealing with refugees there exists international laws and regulations. These also states what kind of human rights exists. Everyone have rights! The vast majority of countries have ratified the UN refugee commission, which has been formed by the international society, UN, and authorities via UN's forums. The commission is an important premise when working with refugees. It is important to know which rights you have as a humanitarian worker, and which rights the refugees have. 

We ask her about how communication within the camp is conducted. She takes Kenya as an example. NRC has been working in the largest refugee camp in the world, Dadaab Kenya, for many years.  Some have the main responsibility for what is going on in the camp, and that is the  authorities. They often ask the international community (e.g. NRC) for help. Wold states that it is then important to establish a good communication and information flow between the ones working in the camp (the different organizations). This communication takes places by either establishing coordination meetings and by other types of mechanisms. These meetings includes the relief organizations working in the camp, and the authorities. The goal is not to make a permanent home for the refugees, but that it is safe when they are in the camps and that they move on (either go home or find another place to live). Living in camps is a temporary life situation. She states the different types of communication; internally between the workers in the camp and communication with the refugees. It is important to establish open transparent coordination mechanisms, in other words ensure good forums where the refugees can communicate and inform the workers in the camp what their needs are. This can only be achieved by recognizing that refugees is not a large mass, but individuals with different needs and different life situations. The humanitarian and authorities try to establish some sort of local elections. This means that the refugees can choose representatives who's job is to be in communication with the primary humanitarian managers in the camp. The reason for this is that it is impossible for the humanitarians to talk to 500 000 people. The communication between the representatives and the managers be done either through meetings, or in an informal manner. Overall, this creates a communication pattern in the refugee camps. Wold states that there a few places without mobile coverage, and that the majority of the refugees have a mobile phone. Mobile phones are used frequently in terms of distribution. Mobile phones are often used as a tool when goods (access to money, food etc.) gets distributed to the refugees. They can "add credit" to their card, and use this as "payment". This an up-and-coming way of doing distribution. Mobile phones are also used to collect information, for example by sending the refugees surveys on their mobile phone. 


In general, Wold states that methods of communication can be via mouth, radio, billboards, data communication, but this all depends on which camp and what is allowed in the camp. The law in the refugee camps depends on the national authorities. In some camps it is allowed to establish a data communication center, but in other camps this is illegal. It is important that when the refugees arrive to a camp that they get informed of the current situation, and what rights they have. The distribution of this information takes places primary by someone called the camp management agency. They have the daily coordination responsibility for what is going to take place in the camp. It must be made clear to the refugees where they can obtain different types of services, and also what is expected of the refugees. It is important that the refugees at an early stage get the opportunity to contribute positively in the camp, or else they can end up with something called "dependency syndrome" (they feel incompetent and get totally dependent on external assistance). 

Another question we asked her is how the refugees get registered in the camps. Here she states the importance of distinguishing between official and unofficial camps. The definition of a camp is that people are gathered together and live there. Registration is done in official camps, and then there are someone who is responsible for the operation of the camp. When refugees are registered they get an ID card. This ID card is very valuable, because it indicates that you, as a refugee, have access to the goods that are available in the camp. The registration procedures can vary, but most often there exists computer systems for the registration.


