\chapter{Refugees and IDPs}
\label{chp:refugeecamps} 
Our initial approach for the thesis was to look into refugee camps, focusing on Norwegian relief organizations. The aim was to get an understanding of how refugee camps are run and how the communication is, both within the camp, and to the outside world. We wanted to look into how the Mesh Potatoes could be utilized to improve the communication. We conducted research trying to get an overview and a general understanding. In addition to this, we carried out interviews with people from Norwegian organizations, namely the Norwegian Refuge Council and CARE. These interviews gave us more information and a better understanding of the area. They also showed us that refugee camps is an extremely comprehensive area, and the differences in the unique camps are enormous depending on country and government law, size of the camp, lifetime of the camp and so on. Some countries have well established Internet and cellular network infrastructure, and people are equipped with smart phones and laptops. In other countries this is far from the case, and their only way of receiving news is by radio or mostly by word-of-mouth. In addition to this, many camps are under strong technological restriction because of country law. 

Since the differences are so big, it is hard to find general information, simply because there are no general information. No camps are equal, and it is hard to compare them. This made it hard to decide in what direction to focus. A research, as the one first intended, would require a close collaboration with a relief organization, and for them to give full focus both to us and our research. It was difficult to create a stable connection and collaboration with one of the Norwegian help organizations. A collaboration like this would require a lot more time than what we had at our disposal. 
  
During our research and work in the first months we could see our report going in a different direction. The area initially chosen became to big to grasp. We decided to direct our focus onto natural disasters, and how a \gls{quick} box can be used to quickly get Internet access and to help coordinate relief work (more information about this in chapter \ref{chp:quickrollout}).
This chapter contains the research we conducted on refugee camps. We will go through some general statistics to get an idea of the life in the camps and the development over the last few years. In addition to this we will present summaries of the interviews with the Norwegian Refugee Council and CARE.

%We therefore chose to co-operate with a smaller organization called Making Change. Making Change is a non profit relief organization where military veterans can use their experience to make a change somewhere in the world.

\section{Definitions}
It is fairy common to think of every person that is displaced as a refugee, but this is not the case. It is important to separate between a refugee and an internally displaced person. 

\subsubsection{Refugee} The definition of a refugee is a person who has been forced to leave his/hers homeland because of, for example, war, violence or persecution. A refugee often has a justifiable fear of persecution for reasons as religion, political opinion, race, nationality or membership in a certain social group. For these reasons they are not able to, or
afraid to, return to their homeland. The leading reasons for refugees fleeing their home country is war and ethnic, religious and tribal violence \cite{refugeeDef}.

\subsubsection{Internally Displaced Person} An internally displaced person (IDP) is a person that has been forced to leave his/her home for some reason and are a refugee in his/her own country. The main distinction between an IDP and a refugee is that the person has not crossed any country borders. Unlike refugees, the IDPs are not protected by any international laws, nor able to receive all types of aid. During the last years the number of IDPs have drastically increased, mostly due to the conflicts between countries \cite{refugeeDef}. 

%\textbf{A stateless person} is someone who does not have citizenship in any country. A citizenship is a legal bond between an individual and the government in that country. 

\section{Statistics}
We have looked at some of the global refugee statistics presented by UNHCR from 2010 and 2012. The following section will enlighten some of these statistics. 

In 2010, the majority of the refugees came from Afghanistan, Iraq and Somalia \cite{UNHCRstat2010}. In both 2010 and 2012 Pakistan was the country which hosted the most refugees. By the end of 2012, 45.2 million people were displaced by force. According to UNHCR this is the largest number seen in 20 years. The report for 2012 show that 55 \% of the registered refugees came from countries affected by war, e.g Syria, Afghanistan, Sudan, Iraq and Somalia. The crisis in Syria has been a major factor for displacement. The whole of 647,000 people have be forced out of the country \cite{UNHCRstat2012}. \tref{tab:refugeestatistics} shows the drastic increase of Syrian refugees from 2010 to 2012. 

An observation made is that the number of Somali refugees has increased from 770,154 refugees in 2010 to 1,136,100 refugees in 2012. There has been an ongoing conflict in Somalia ever since the Siade Barre regime collapsed in 1991. This conflict have resulted in the displacement of many Somalis. The number of displaced persons is constantly changing. In 2011-2012 there was a famine in Somalia, and this caused not only many deaths, but the displacement of many people \cite{somalia}. This is one of the reasons for the increase of Somali refugees from 2010 to 2012. 

In contrary to the increase of Somali refugees between 2010 and 2012, the number of Iraqi refugees had decreased from 1,683,579 refugees in 2010 to 746,400 refugees in 2012. Before the Syrian civil war started in 2011, there were many Iraqi refugees who had fled to Syria due to the invasion led by the U.S. When the Syrian civil war began, the situation reversed. Many Syrian people sought shelter in Iraq, and many of the Iraqi refugees returned to their homeland. This is one reason for the drastic decrease of Iraqi refugees from 2010 to 2012. Although many Iraqi refugees went back to Iraq, they remained displaced. This situation has brought the number of Iraqi IDPs up to approximately 2.8 million \citep{iraq}. 

\begin{center}
\begin{table}[!ht]
\caption{\label{tab:refugeestatistics}Refugee statistics - Comparing 2010 and 2012 \cite{UNHCRstat2010,UNHCRstat2012}}
    \begin{tabular}{ | p{6cm} | l | l | l |}
    \hline
    \textbf{Information} & \textbf{2010} & \textbf{2012} & \textbf{\% variation}\\ 
    \hline
    Number of people forcibly displaced worldwide & 43.7 million & 45.2 million & + 3.4\% \\ 
    \hline
    Number of refugees from Afghanistan & 3,054,709 & 2,585,600 & $\div$ 18\% \\ 
    \hline
    Number of refugees from Somalia & 770,154 & 1,136,100 & + 47.5\%\\ 
    \hline
    Number of refugees from Iraq & 1,683,579 & 746,400 & $\div$ 125\% \\ 
    \hline
     Number of refugees from Syria & 18,452 & 728,500 & + 3848\% \\ 
    \hline
    Number of refugees hosted by Pakistan & 1,900,621 & 1,638,500 & $\div$ 13.8\% \\
    \hline
    Percentage of refugees that are female & 47\% & 48\% & + 1\%\\
	\hline
	Percentage of refugees that are children (below 18) & 47\% & 46\% & $\div$ 1\%\\
	\hline
	Number of individual asylum applications lodged by unaccompanied or separated children & 15,500 & 21,300 & + 37.4\% \\
	\hline
    \end{tabular}
   \end{table}
\end{center}


\section{Interview with Norwegian Refugee Council}
\label{sec:interviewnrc}
We had a Skype interview March 12, 2014, with Katrine Wold from the \gls{nrc}. The aim of the interview was to hear about her work in refugee camps and how the situation in the refugee camps are today, with main focus on means of communication.
Katrine Wold has been working for \gls{nrc} for many years, and also has a background from United Nations (UN). She has worked in emergency and crisis situations abroad. She is specialized in camp management and coordination. In recent years she has been responsible for education, and have had the main focus on youth. We asked her which refugee camps \gls{nrc} is working in, but she could not give us a clear answer on this question. The reason for this is that \gls{nrc} works in over 24 countries and have, as of 2013, reached out to 4.4 million people. She makes it clear that there is a difference between internally displaced persons (IDPs) and refugees. An official refugee must cross a boarder, or else you are internally displaced. \gls{nrc} works both with refugees and IDPs, and also with people who are affected by having refugees in their local area. \gls{nrc} does not only help with operational issues in the camps, but they mainly offer services that the refugees need. When dealing with refugees, there exists international laws and regulations. These also states what kind of human rights exists. Everyone have rights! The vast majority of countries have acknowledged the UN refugee commission, which has been formed by the international society, UN, and authorities via UN's forums. The commission is an important premise when working with refugees. It is important to know which rights you have as a humanitarian worker, and which rights the refugees have. 

We ask her how communication within the camp is conducted. She takes Kenya as an example. \gls{nrc} has been working in the largest refugee camp in the world, Dadaab Kenya, for many years. The authorities have the main responsibility for what is going on in the camp. They often ask the international community (e.g. \gls{nrc}) for help. Wold states that it is important to establish a good communication and information flow between the different organizations working together in the camp. This communication takes place by, for example, establishing coordination meetings. These meetings includes the relief organizations working in the camp, and the authorities. The goal is not to make a permanent home for the refugees, but to make the camps a safe place to live temporarily, and to help them move on (either go home or find another place to live). Living in camps is a temporary life situation. She states the different types of communication; internally between the workers in the camp and communication with the refugees. It is important to establish transparent coordination mechanisms, in other words ensure good forums where the refugees can communicate and inform the workers in the camp what their needs are. This can only be achieved by recognizing that refugees are not a large mass, but individuals with different needs and with different life situations. The humanitarians and authorities try to establish some sort of local elections. This means that the refugees can choose representatives who's job is to be in communication with the primary humanitarian managers in the camp. The reason for this is that it is impossible for the humanitarians to talk to 500 000 people. The communication between the representatives and the managers must be done either through meetings, or in an informal manner. Overall, this creates a communication pattern in the refugee camps. Wold states that there are few places without mobile coverage, and that the majority of the refugees own a mobile phone. Mobile phones are often used as a tool when goods (access to money, food etc.) are distributed to the refugees. They can "add credit" to their card, and use this as payment. This an up-and-coming way of doing distribution. Mobile phones are also used to collect information, for example by sending the refugees surveys on their mobile phone. 


In general, Wold states that methods of communication can be via mouth, radio, billboards, data communication, but this all depends on which camp and what is allowed in the camp. The law in the refugee camps depends on the national authorities. In some camps it is allowed to establish a data communication center, but in other camps this is illegal. It is important that the refugees get informed of the current situation upon arrival, and of what rights they have. The distribution of this information takes place primarily by someone called the camp management agency. They have the daily coordination responsibility for activities taking place in the camp. It must be made clear to the refugees where they can obtain different types of services, and also what is expected of the refugees. It is important that the refugees at an early stage get the opportunity to contribute positively in the camp, or else they can end up with something called "dependency syndrome" (they feel incompetent and get totally dependent on external assistance). 

Another question we asked her is how the refugees get registered in the camps. Here she states the importance of distinguishing between official and unofficial camps. The definition of a camp is that people are gathered together and live there. Registration is only done in official camps. When refugees are registered they get an ID card. This ID card is very valuable, because it indicates that you, as a refugee, have access to the goods that are available in the camp. The registration procedures can vary, but most often there exists computer systems for the registration.



\section{Interview with CARE - Dadaab Refugee Camp}
\label{sec:interviewcare}
We got in contact with Mary Muia from CARE. She is a program assistant at CARE International in Dadaab, Kenya. We sent her a questionnaire with questions about Dadaab refugee camp, with focus on means of communication. The following section contains information both from different articles and from the answers of the questionnaire. See Appendix \ref{chp:appendixA} for the full questionnaire. 

Dadaab is the largest refugee camp in the world, and is located in Kenya, Africa \cite{dadaab}. It was created in 1991 by the government of Kenya and UNHCR to host Somali refugees displaced by civil war \cite{dadaabcare}. Over the years, the camp has also hosted other nationalities, from the Horn of Africa, the Great Lakes and East African regions. These people constitute less than two percent of the camp population. In April, 2013, there were 423,496 registered refugees in the Dadaab camp. 51 \% of these were female and 58 \% were younger than 18 years old. Also in 2013, UNHCR and its partners decided to conduct a verification exercise to ascertain the current population. The reason for this was that many of those who had arrived in 2011 due to the famine had returned home. As of February, 2014, the current population stands at 369,294. The lead agency for this camp is the UN High Commission for Refugees (UNHCR) \cite{dadaab}. In addition to UNHCR, major international humanitarian agencies like CARE, Save the Children and the International Rescue Committee are active helpers in the Dadaab refugee camp. These agencies provide the refugees with critical services (e.g. food, housing, sanitation and medical help). This is an extremely challenging task in refugee camps, especially when they reach this size. During the recent years, the terror group Al Shabaad (Somali-based) have intensified their misdeeds in, and around, the Dadaab refugee camp. This has made the situation even tougher for the refugees and the relief agencies. 
Muia states that the biggest challenges in the camp are lack of space to accommodate everyone, and the lack of funding to take care of all the needs of the refugees. Another challenge is the language barrier between the humanitarian staff and the refugees. Many of the staff members neither speak nor understand the Somali language, and as many as 95.6\% of the refugees are Somali. 

Muia explains how the registration process is handled; when a new refugee enters the camp, the refugee reports to a UNHCR reception desk. There the refugee is given a temporary registration, while pending full registration. Upon arrival, the refugees are given information about available services, and which agency is handling what service. Immunizations, medical attention, emergency food supply, tarpaulins, sleeping mats, jerrycans for fetching water and kitchen sets are issued on arrival. This is to help them start their new lives in the camp. 

To improve the situation in Dadaab, communication is crucial. In 2011, a group consisting of people from NetHope, Inveneo and the USAID Global Broadband and Innovations Program gathered to discuss ways to improve the means of communication in Dadaab \cite{dadaab}. NetHope is a consortium of over 30 international \glspl{ngo} \cite{nethope}. NetHope works with improving connectivity, with the help of information technology, among relief agencies. The aim of this project, called DadaabConnect, was to bring forward more reliable Internet, and find ways for agencies to communicate better internally \cite{dadaab}. The group put together teams that travelled to Kenya to investigate the conditions in the refugee camps, and to find out what they could implement. It was clear from the feedback they got that a better communication system was needed, and that it would make the humanitarian work much easier. It would improve both the coordination and the security in the camp. Improvements of these aspects gives the humanitarian agencies better working conditions, and makes it easier for them to help the refugees with critical services. Inveneo started working with Cisco's Tactical Operations (TacOps) to install and configure a local high-speed network \cite{dadaabinveneo}. They also engaged in a partnership with a local Kenyan mobile and landline telecommunications service provider called Orange. The reason for this was that they wanted to extend the Dadaab compound with new data services. This could be done by using Inveneo's long-distance Wi-Fi solutions. The data services that were added included services requested from the Dadaab aid community. "DadaabNet", a high-speed network, was created in cooperation between Inveneo and TacOps. This network connected the NGOs locally, and made it possible for the agencies to easier communicate internally (VoIP telephony, file sharing etc.). Following this, in March 2012, they started the training of technicians. These technicians were people from Orange, from the technical staff of the NGOs and from Inveneo's staff. The training took place both in classrooms and in the field, in order to give the technicians a wide understanding. The results from DadaabConnect has been great. The humanitarian agencies has gotten better working conditions, due to the improvements in means of communication. Other positive outcomes is that the network is more reliable and cost effective. 

We asked questions about means of communication within the camp, and with the outside world. Muia did not specifically mention the project described above. She states that CARE as an organization has invested in communication systems in cooperation with ISPs in the capital city of Kenya, Nairobi. Through this cooperation, the camp staff are assured to get Internet access for both official and social purposes. Several Kenyan telecommunications companies have set up equipment in the camp area, and the camps are therefore provided with access to mobile communication and Internet. Unfortunately, Internet and telephone service outages are fairly common. In addition to mobile communication and the Internet, there are radio station services and access to digital television. CARE use telephone services to reach out to refugee staff. 50\% of the refugees have access to mobile phone services. Posters and radio are also commonly used to reach out. Word-of-mouth (e.g. over speakers) is also a communication technique employed. There are two main telecommunication providers in Dadaad, hence little competition. The lack of competition makes the prices higher. We asked Muia how the refugees can afford having their own mobile phone, when the costs are so high. She says that many refugees have been in the camps for a long time, and therefore have had the time to establish small businesses which gives them some profit. Others get money sent from their relatives. 


\section{Life in Camps for Refugee Women}
In this section we will shortly present some of the most relevant answers found in research done by Mari Maasilta, a Swedish post doctoral researcher. She has looked at the use of oral and mediated communication by women living in refugee camps in Eastern-Africa. 

Women are in general in a more vulnerable position when living in a  camp, especially if they are single mothers. They may be solely responsible for taking care of the children, in addition to sick and elderly family members, maintaining the household, preparing food, acquire water, and securing firewood. Collecting firewood for cooking is a necessity, but it forces women  to walk far away, hence making them vulnerable for sexual assault. They often have to turn to prostitution and other unhealthy and dangerous means in order to survive \cite{womenRefugee}. 

The means of communication vary greatly in the different camps. Some have Internet connection and satellite TV, other barely have access to a radio. Even though radios are the most common media for communication, it is not given that all citizens in a camp have access to one. Often people gather around the few radios that exists in a camp. One issue that limits the usage is the batteries, since they are very expensive and hard to acquire. The use of cell phones are increasing. Even though the prises are extremely high it does not stop people from calling relatives in Europe and other places in the world \cite{womenRefugee}.

Information walls and word-of-mouth is often used in order to spread practical information within the camp and about camp activities. Word-of-mouth is also used in order to retrieve information about the world outside the camp. People visiting the camp were used as sources for information. Earlier studies have shown that social connections with neighbours works as an important medium to transport information, resources and services between individuals. This kind of networking has been used to find lost family members in big camps, as well as get financial help from abroad \cite{womenRefugee}.  


