\chapter{Refugee Camps}
\label{chp:refugeecamps} 

\section{Refugee Camps}
%Generell fakta

Women are in general in a more vulnerable position when living in a  camp, especially if they are single mothers. They may be solely responsible for taking care of the children in addition to sick and elderly family members, maintaining the household, preparing food, acquire water, and securing firewood. Collecting firewood for cooking is a necessity, but it forces women  to walk far away, hence making them vulnerable for sexual assault. they have to turn to sex and other unhealthy and dangerous means in order to survive. \cite{womenRefugee} 


The means of communication vary greatly in the different camps. Some have internet connection and satellite TV, other barely have access to a radio. Even though radios are the most common media for communication, it is not given that all citizens in a camp have access to one. Often people would gather around the few radios that exists in a camp. One issue that limits the usage is the batteries. They are very expensive and hard to acquire. The women were interested in news regarding their place of origin. 

Information walls and word of mouth is often used in order to spread practical information within the camp and about camp activities. Word of mouth is also used in order to retrieve information about the world outside the camp. People visiting the camp were used as sources for information. Earlier studies have shown that social connections with neighbours works as an important medium to transport information, resources and services between individuals. These kind of networking have been used to find lost family members in big camps, as well as get financial help from abroad.   
\cite{womenRefugee}  



\paragraph{Cell Phone}
The use of cell phones are increasing. Even though the prises are extremely high it does not stop people from calling relatives in Europe and other places in the world. 
\cite{womenRefugee} 

\section{Dadaab Refugee Camp}
Dadaab is the largest refugee camp in the world, and is located in Daadaab, Kenya \cite{dadaab}. It was created in 1991 \cite{dadaabcare}. In April, 2013, there were 423,496 registered refugees in the Dadaab camps. 51 \% of these were female and 58 \% were younger than 18 years old. The lead agency for this camp is the UN High Commission for Refugees (UNHCR) \cite{dadaab}. In addition to UNHCR, major international humanitarian agencies like Care, Save the Children and the International Rescue Committee  are active helpers in the Dadaab refugee camp. These agencies provide the refugees with critical services (e.g. food, housing, sanitation and medical help). This is an extremely challenging task in refugee camps, especially when they reach this size. During the recent years, the terror group Al Shabaad (Somali-based) have intensified their misdeeds in and around the Dadaab refugee camps. This has made the situation even tougher for the refugees and the relief agencies. 

\paragraph{Means of Communication}
To improve the situation in Dadaab, communication is crucial. In 2011, a group consisting of people from NetHope, Inveneo and the USAID Global Broadband and Innovations Program gathered to discuss ways to improve the means of communication in Dadaab \cite{dadaab}. NetHope is a consortium of over 30 international Non-Governmental Organizations (NGOs) \cite{nethope}. NetHope works with improving connectivity, with the help of information technology, among relief agencies. The aim of this project, called DadaabConnect, was to bring forward more reliable Internet, and find ways for agencies to communicate better internally \cite{dadaab}. The group put together teams that travelled to Kenya to investigate the conditions in the refugee camps, and to find out what they could implement. It was clear from the feedback they got that a better communication system was needed, and that it would make the humanitarian work much easier. It would improve the coordination and the security in the camps. Improvements of these aspects gives the humanitarian agencies better working conditions, and makes it easier for them to help the refugees with critical services. Inveneo started working with Cisco's Tactical Operations (TacOps) to install and configure a local high-speed network \cite{dadaabinveneo}. They also entered a partnership with a local Kenyan mobile and landline telecommunications service provider called Orange. The reason for this was that they wanted to extend the Dadaab compound with new data services. This could be done by using Inveneo's long-distance Wi-Fi solutions. The data services that were added included services requested from the Dadaab aid community. "DadaabNet", a high-speed network, was created in cooperation between Inveneo and TacOps. This network connected the NGOs locally, and made it possible for the agencies to easier communicate internally (VoIP telephony, file sharing etc.). Following this, in March 2012, they started the training of technicians. These technicians were people from Orange, from the technical staff of the NGOs and from Inveneo's staff. The training took place both in classrooms and in the field, in order to give the technicians a wide understanding. The results from DadaabConnect has been great. The humanitarian agencies has gotten better working conditions, due to the improvements in means of communication. Other positive outcomes is that the network is more reliable and cost effective. 

%As the new network architecture is tried and proven to be more reliable and cost effective, it will be extended to the general population via sustainable outreach community centers that support learning, resettlement and economic empowerment. The Dadaab Connect project is funded by Inveneo’s Broadband for Good Program, Cisco, Microsoft, NetHope, Craig Newmark, the Orr Family Foundation, UNHCR, and USAID’s Global Broadband Innovations Program.


\section{Challenges}

\subsection{General Challenges}

\subsection{Communication Challenges}

