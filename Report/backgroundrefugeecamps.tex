\chapter{Refugee Camps}
\label{chp:refugeecamps} 

\section{Refugee Camps}
%Generell fakta

Women are in general in a more vulnerable position when living in a  camp, especially if they are single mothers. They may be solely responsible for taking care of the children in addition to sick and elderly family members, maintaining the household, preparing food, acquire water, and securing firewood. Collecting firewood for cooking is a necessity, but it forces women  to walk far away, hence making them vulnerable for sexual assault. they have to turn to sex and other unhealthy and dangerous means in order to survive. \cite{womenRefugee} 


The means of communication vary greatly in the different camps. Some have internet connection and satellite TV, other barely have access to a radio. Even though radios are the most common media for communication, it is not given that all citizens in a camp have access to one. Often people would gather around the few radios that exists in a camp. One issue that limits the usage is the batteries. They are very expensive and hard to acquire. The women were interested in news regarding their place of origin. 
Information walls and word of mouth is often used in order to spread practical information within the camp and about camp activities. Word of mouth is also used in order to retrieve information about the world outside the camp. People visiting the camp were used as sources for information. Earlier studies have shown that social connections with neighbours works as an important medium to transport information, resources and services between individuals. These kind of networking have been used to find lost family members in big camps, as well as get financial help from abroad.   
\cite{womenRefugee}  

\paragraph{Cell Phone}
The use of cell phones are increasing. Even though the prises are extremely high it does not stop people from calling relatives in Europe and other places in the world. 
\cite{womenRefugee} 

\section{Dadaab Refugee Camp}
%Når ble den oppretta?
Dadaab is the largest refugee camp in the world, and is located in Daadaab, Kenya \cite{dadaab}. There are as many as 500,000 registered refugees in this camp. The lead agency for this camp is the UN High Commission for Refugees (UNHCR). In addition to UNHCR, major international humanitarian agencies like Care, Save the Children and the International Rescue Committee  are active helpers in the Dadaab refugee camp. These agencies provide the refugees with critical services (e.g. food, housing, sanitation and medical help). This is an extremely challenging task in refugee camps, especially when they reach this size. 

\paragraph{Al Shabaab}
Al Shabaad is a Somali-based terrorist group. Al Shabaab started doing miscellaneous misdeeds both in the Dadaab refugee camp, and around the camp.  
%Når begynte de med dette?


\section{Challenges}

\subsection{General Challenges}

\subsection{Communication Challenges}

