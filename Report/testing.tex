\chapter{Testing the QUICK Box}
\label{chp:test} 

In order to assure the quality of the manuals and the \gls{quick} box, we conducted a test iteration. Testing is one of the most important parts of the process when developing a product. Using test persons with different technical background and knowledge, as well as different gender and age, provided us with valuable feedback and comments on issues that we in advance did not foresee. 

\begin{center}
\begin{table}[h!]
\caption{\label{tab:testpersons}Key facts about the test participants}
    \begin{tabular}{ | l | l | l | l |}
    \hline
    \textbf{Test person} & \textbf{Gender} & \textbf{Age} & \textbf{Technical knowledge} \\ 
    \hline
    Test person 1 &  Male & 28 & High\\ 
    \hline
    Test person 2 &  Female & 25 & High\\  
    \hline
    Test person 3 &  Female & 22 & Low\\  
    \hline
    Test person 4 &  Male & 21 & Low/medium\\  
    \hline
    \end{tabular}
   \end{table}
\end{center}


\section{Test Procedure}
The main focus when performing the tests was to assure the quality of the manuals we had created. We wanted to find out if the manuals were intuitive and user-friendly.

The tests were performed in our office, since the focus was on the manuals and not on the concept of the \gls{quick} box. The test persons were given the different manuals, an \gls{mp}, an Ethernet cable, a USB-stick containing the script and a stationary PC running Linux Ubuntu with WiFi connection. We gave them a short introduction to what our project consists of. After this introduction, the test persons started following the manuals, one at a time. In real life there will most likely only be necessary to run through one of the manuals. Every test person ran through all the manuals, in order for us to get as much feedback as possible. This gave the test persons the opportunity to compare the different manuals. While the test persons were running through the manuals, we observed their actions. The test persons were allowed to ask questions, although it was desirable that the test persons tried themselves before asking. If we observed that the test persons performed mistakes, caused by lack of information or poor information in the manual, we stopped the test persons and guided them towards correcting the mistakes. After a manual was completed, the test persons assured that they were able to connect to the Internet via the \gls{mp}, and were asked to answer a few questions regarding their experience. The questions concerned their previous use, and knowledge, of Linux and how this affected their test experience, how easy they felt it was to follow the manual, and if they had any suggested improvements.
While the questions were asked, the \gls{mp} was rebooted and set up ready for the next manual. The same procedure was followed for all three  manuals. 

The test was first performed on test person 1 which is considered a person with high technical knowledge. This test showed that the manuals were lacking a lot of crucial information. We chose to correct the shortcomings before conducting the tests on the remaining test persons, since these shortcomings went on the expense of the test's purpose. 

\section{Test Results}
We will go through the results for each of the manuals and highlight the most interesting findings, before comparing them. 

In the results presented we distinguish between test persons with technical and "non-technical" knowledge. The technical test persons are people with great knowledge about IT and programming. They have both studied computer science or similar. The "non-technical" test persons does not have an IT background. They are everyday users of technical devices and social media, but does not contain a deeper technical understanding or interest. 
 
\subsection{Manual 1: Manual for Connecting the MP02 Directly to Cabled Internet}
The complete manual can be found in section \ref{subsec:cabledInternet}.

After test person 1 tested this manual, we did a lot of changes to the language and to how the steps were described. The figures were changed out with more descriptive ones and placed closer to the respective steps in order to make it more visual for the user. 

This manual was considered the easiest by all test persons. There were little confusion regarding the steps described. The test persons expressed that they liked the pictures and felt that it worked as an assurance that they were conducting the steps correctly. Test person 3 and test person 4 both expressed some confusion regarding step 3 where the user is asked to write in terminal-commands. Originally, there were no information stating that the user had to open the terminal, or on how to open the terminal. Due to this feedback, information regarding this has been added to the manual. The last step also caused some confusion. Originally, the step described that the login information could be found in the "Get started -How to Use the QUICK Box"-document. This document was provided to the test persons, but non of them even looked at it or asked for it. The text in this step has also been improved to avoid confusion. 

\subsection{Manual 2: Manual for Connecting the MP02 to the Internet via PC Getting WiFi from Landline or Cellular Network}
The complete manual can be found in section \ref{subsec:internetviaPC}.

Test person 1 expressed some confusion while running through this manual. One obstacle test person 1 pointed out was the use of the word \textit{octet}. Test person 1 is considered as a person with high technical knowledge, but were still unsure about the meaning of this word. In the manual we added more information explaining that with "the last octet" we mean the last number of the IP address. Another step that confused test person 1 was step 5. In this step the user is asked to enter some terminal-commands where the user has to change two of the variables. It was unclear to test person 1 what to write and how to find out if the variables were correct or not. We changed this text to include more information about which variables should be changed and on how to find these. Test person 1 stopped up when receiving an "error-message" after entering the command "route del default" in step 6. This command deletes the default route, if one exists. If no default route is set, an "error-message" will be displayed. Test person 1 was unsure if he had done something wrong. We added information stating why this "error-message" appears and that it can be ignored. 

Most steps in this manual are executed in the terminal. Two of the test persons had no previous experience with Linux or the use of the terminal. On this area we saw a clear difference between the technical and "non-technical" test persons. The technical test persons had no problem finding the terminal and immediately started entering the commands described in the manual. The "non-technical" test persons expressed hesitation and confusion around where to find the terminal and on how to write commands. Test person 4 started writing a command with the "\$". This symbol is used to indicate that the following is a command. Test person 3 started writing a new command before the last one was "finished". This emphasized the lack of Linux knowledge and Linux-terminal experience.   

In step 4, the user is asked to open the web browser and type in the MP's IP address in order to make sure that the PC has established contact with the \gls{mp}. Test person 3 typed in "192.168.1.x", which is exactly what is stated in the manual. Previously in the manual, it is described that "x" has to be replaced with the last number of the IP address. The same mistake happened in step 6, where test person 3 also forgot to replace the "x"-value with the last number of the IP address. We have now added this information in the given steps as well. 

Even though we made some changes to step 5, after test person 1's feedback, this step still caused a lot of confusion for the remaining test persons. All started typing the command exactly as stated in the manual, even though it was described in the step that some parameters had to be changed. We rewrote this text in order to avoid any more confusion. Only test person 2 understood that the parameter "<WIRELESS INTERFACE>" had to be changed. The correct alteration is "wlan0", but test person 2 typed "<wlan0>", meaning she did not remove the "crocodile symbols".

The last step also caused some confusion. Originally, the step described that login information could be found in the "Get started - How to Use the QUICK Box"-document. This document was also provided to the test persons, but non of them even looked at it or asked for it. The text in this step has also been improved to avoid this confusion.

\subsection{Manual 3: Script providing Internet access via PC}
The complete manual can be found in section \ref{subsec:internetviaPC} under the headline "Script providing Internet access via PC".

While testing this manual, test person 1 pointed out that the manual was missing a lot of necessary information. Test person 1 stopped up at step 1, where the manual did not describe which port on the \gls{mp} the Ethernet cable should be plugged into. Nor did the manual contain any information regarding that the user had to open the terminal and on how to open it. The user was provided with a USB-stick containing the script, but this was not stated in the manual, nor that the USB-stick has to be plugged in before entering the given commands. After the script started, and the first pop-up window appeared, test person 1 immediately looked at the manual for further instructions, although these were provided in the pop-up window. The manual did not contain any information about what to do after the script started running. Test person 1 also expressed that it was difficult to understand what was meant by "wireless interface". Test person 1 was one of the technical people participating in our test and were able to figure out what the wireless interface was on his own. There were no confusion regarding the second pop-up window, but he pointed out that the text could be structured in a different way to better emphasize the important content.
Test person 1 also pointed out that it was positive that the commands were written in a separate font, since this made it easier to understand. 

We improved the manual based on the feedback from test person 1. Test person 3 were one of the "non-technical" test persons, and got confused when the first pop-up window appeared. Even though we made it very clear in the manual that it is important for the users to carefully read the information given in the script, test person 3 immediately looked at the manual when she saw the first pop-up. The script asks the user to enter two variables in the terminal: the last number of the IP address and the name of the wireless interface. Test person 3 entered just one variable before pressing enter, while test person 2 entered the whole IP address. 
Several test persons expressed confusion regarding the pop-up windows. Originally, the pop-up window appeared before the browser opened, which means that the test persons did not notice it. We have now changed this so that the browser opens first, then the pop-up window. 

The last step also caused some confusion. Originally, the step described that login information could be found in the "Get started -How to Use the QUICK Box"-document. This document was also provided to the test persons, but non of them even looked at it or asked for it. The text in this step has also been improved to avoid this confusion.

\subsection{Summary of the Test Results}
Test person 1 was the first person to test our manuals. He found a lot of mistakes and possible improvements. After the manuals were updated and run on the remaining test persons there were still steps that was considered confusing. The areas of confusion were mainly around the same steps. All test persons had difficulties understanding what they were supposed to do in step 5 of manual 2, concerning changing the port names. All test persons were also confused about the log-in information in all the three manuals. The test persons had most difficulties with manual 2, and the least difficulties with manual 1. Test person 3 and test person 4 preferred manual 1, because it was the easiest and did not involve much writing. Test person 1 and test person 2, on the other hand, preferred manual 3, script. 

The testing process went according to plan, and all the test persons were satisfied. We did not encounter any obstacles of significance, and collected a lot of valuable feedback.  