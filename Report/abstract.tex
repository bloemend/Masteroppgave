\renewcommand{\abstractname}{Abstract}
\begin{abstract}

The Village Telco organization aims to provide affordable communication by means of data and voice services where no other companies are able or willing to do so. Village Telco provides a “plug-and-play” solution with low cost voice and data service. This solution is delivered using an inexpensive fixed mesh WiFi delivery system called Mesh Potato. Village Telco’s solution can be applied anywhere in the world where people wish to take control of their own communications infrastructure. Mesh Potato networks can be deployed either as a stand-alone solution or as an extension to existing technologies. Village Telco’s solution has been deployed in several countries around the world. The installed communities vary from ten, to several hundreds of Mesh Potatoes. 

We directed our studies toward the use of Mesh Potatoes in mobile situations. We have looked at different scenarios covering everything from emergency situations and natural disasters, to festivals and temporary refugee camps. Keith Williamson, a Village Telco volunteer, has created a "go box" using the first generation of the Mesh Potato. We wanted to take his solution further and developed a mobile and stand-alone kit that could be used in all the different scenarios mentioned. The prototype we have developed is slightly different from the "go box". We have used the second generation Mesh Potato and the box includes everything necessary for a quick roll-out; solar panel, battery, different cables and easy to use manuals for configuration. 

We established that many of the descriptions found on the Village Telco wiki were not clear and difficult to use. Hence a big part of our work consisted in simplifying these descriptions. We created manuals for connecting the Mesh Potato to different uplinks in order to provide Internet access to the network. Four test persons, both with and without technical knowledge, tested the manuals. The test process gave us valuable feedback, which led to improvements in the manuals. We have looked at the process of roll-out, and stated possible simplifications and improvements in order to make the roll-out as quick as possible. Hence we have named our solution the \gls{quick} box.

We believe that our work and theoretical research has contributed to and enriched the Village Telco community. Our prototype can be further developed and introduced to relief organizations as well as to others who express an interest in a mobile communications kit. 


\end{abstract}
