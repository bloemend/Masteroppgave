\subsubsection{Script providing Internet via PC}

An alternative, and easier method in order to provide Internet to the \gls{mp} via a PC with WiFi is running a script. A script is a list of commands that can be executed with less user interactions to automate a process. The script is provided on a USB-stick that is included in the \gls{quick} box. 

Some easy steps must be conducted to run the script:
\begin{enumerate}
\item Connect the MP (make sure the MP is powered up) to the PC, running Linux, with an Ethernet cable. The Ethernet cable must be plugged into the LAN-port on the MP.
\item Plug the provided USB-stick into the PC.  
\item Open the terminal on your PC by pressing "Ctrl+Alt+t". 
\item Execute the following commands to enter the right directory (where the script is located):
\noindent
\begin{lstlisting}[language=bash]
  $ sudo su
  $ cd /media/MP-USB
\end{lstlisting}
\item Execute the following command to run the script (Follow the descriptions provided in the script from now on. This information will both be given in the terminal window and as pop-up windows. Make sure you read these carefully):
\noindent
\begin{lstlisting}[language=bash]
  $ sh scriptviaPC.sh
\end{lstlisting}
\item Internet should now be available in the mesh network. You can test this by connecting to the \gls{mp} from, for example, your smart phone or a PC. 
\begin{itemize}
\item The name of the network: \textbf{MP2_21}
\item The password is: \textbf{potato-potato}
\end{itemize}
Both the name and the password can be changed in the web interface. 
\end{enumerate}