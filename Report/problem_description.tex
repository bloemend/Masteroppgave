\begin{titlingpage}

\noindent
\begin{tabular}{@{}p{4cm}l}
\textbf{Title:} 	& \thetitle \\
\textbf{Student:}	& \theauthor \\
\end{tabular}

\vspace{4ex}
\noindent\textbf{Problem description: }
Village Telco is an organization that aims to provide the solution for local communities can deploy data and voice services where no other companies can, or are willing to do so. Village Telco provides a “plug-and-play” solution with low cost voice and data service. While designed for the developing world, Village Telco’s solution can be applied anywhere where people wish to take control of their own telephone infrastructure.

This solution is delivered using an inexpensive fixed mesh WiFi delivery system called the Mesh Potato. The MeshPotato unit is based on the open-source operating system, OpenWRT. Open Source telephony software combined with the latest wireless networking technology creates the potential for people to operate their own community phone systems. Mesh Potato networks have no dependence on existing telecom infrastructure, and can relatively easily be deployed anywhere in the world. It can either be deployed as a stand-alone solution or as an extension to existing technologies. Village Telco’s solution has been deployed in several countries around the world: from East-Timor and Nepal in Asia to several African and South American countries. The installed bases vary from 10 to several hundreds of Mesh Potatoes. 

An area that has not been fully explored is the use of Mesh Potatoes in emergency situations, like natural disasters, post-conflict situations, etc. Another area to be considered is the use of Mesh Potatoes in refugee camps, where many people quickly gather in a new location. In both situations, the need for communication is essential. Key factors of usage are quick roll-out and usability. Easy to use communication is extremely important in crisis situations, both communication within the camp and outgoing communication with the rest of the world. It is important that all affected have easy access to helpful information, as this could mean the difference between life and death in some situations. In refugee camps with thousands of people, registering and reuniting people can be a difficult task to solve. Communication technology, like the Mesh Potato, could be revolutionary in situations like these. 

We will look into how communication is handled by Norwegian emergency relief organizations today, what tools they are using, and if their way of communication could improve with the Mesh Potato. In addition to this, we will look into other existing tools, and explore the possibilities to combine them with the Mesh Potato for a better product.

\vspace{2ex}

\noindent \Blindtext[2][1]
\vspace{6ex}

\noindent
\begin{tabular}{@{}p{4cm}l}
\textbf{Responsible professor:} 	& \theprofessor \\
\textbf{Supervisor:}			& \thesupervisor \\
\end{tabular}

\end{titlingpage}