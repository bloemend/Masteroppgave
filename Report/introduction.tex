\chapter{Introduction}
\label{chp:introduction} 

\section{Motivation}
As of now the Mesh Potato has mainly been permanently deployed in small villages where the existing telecommunication systems are limited, non existent or too expensive. There are many scenarios where there is need for a solution that easily, and fast can provide people with telephone communication and Internet, both within a community, and with the outside world. These scenarios span from natural disasters, post-conflict situations, temporary refugee camps and IDP (internally displaced person) camps, to the use at festivals, when a mobile tower is non-functioning, or during a blackout. 

We hope to expand the potential of the Mesh Potato through our portable solution. We want to make it quick and easy to deploy, thus making it more useable in emergency situations. This does not only benefit the locals, but also makes the job easier for relief organizations. 
We want to provide communication where there are none,  and believe that with the “emergency box” time would be spared and lives can be saved.

An area that has not been fully explored is the use of Mesh Potatoes in emergency situations, like natural disasters, post-conflict situations, etc. Another area to be considered is the use of Mesh Potatoes in refugee camps, where many people quickly gather in a new location. In both situations, the need for communication is essential. Key factors of usage are quick roll-out and usability. Easy to use communication is extremely important in crisis situations, both communication within the camp and outgoing communication with the rest of the world. It is important that all affected have easy access to helpful information, as this could mean the difference between life and death in some situations. In refugee camps with thousands of people, registering and reuniting people can be a difficult task to solve. Communication technology, like the Mesh Potato, could be revolutionary in situations like these. 


\section{Problem Description}
As our main problem description shows, the initial approach on our thesis was to look info refugee camps and how the Mesh Potatoes could be utilized in these situations. We started contacting different Norwegian relief organizations, but found it hard to establish a good connection with any of them. We also saw that the field was enormous and to much for us to grasp with the limited amount of time that we had available. A deciding factor was also that we saw the need to visit a camp in order to understand how the refugee camps works, and what the need in forms of communication would be. Everyone that we were in contact with said that no two camps are the same or run in the same way. Also the camps are often run by the local government with help from the different relief organizations. Different countries have different laws and regulations, and these also have to be taken into consideration. Without being able to early in the process establish a cooperation with a relief organization, we decided to direct our focus in a slightly different direction. We therefore chose to look into the use of the Mesh Potato in different scenarios with the focus on quick roll-out and providing Internet.

Our main focus is to provide the people with Internet access, since it is crucial to have the possibility to communicate with the outside world during an emergency situation. In order to get Internet into the mesh network formed by the Mesh Potatoes, at least one of the Mesh Potatoes must be connected to Internet. Which type of access network that is available depends on the location. Some places there might exist stable landlines, other places not. Other options could then be to use satellite or cellular networks to provide the network with Internet.  

Our idea is to make an "emergency box" that consist of a Mesh Potato, a telephone, rechargeable battery, on/off switch and a solar panel to charge the battery. All this will be contained inside a robust and waterproof suitcase. All packed together and ready to go in any situation, at any time, anywhere in the world. 

Based on our motivation we researched and conducted a study in order to answer the following research questions: 

\begin{itemize}
\item How are the Mesh Potatoes set up? 
\item How can an emergency box be developed? What components, set-ups and configurations are necessary?
\item What kind of up-links can we connect the Mesh Potato to? And how can this be done easily? 
\item How make the roll-out process as quick as possible? What measures can we do in advance to make it as easy and fast as possible to connect the go-box to and internet connection?
\item In what kind of situations could there be a need for a portable emergency box? What are the need in the different situations?
\end{itemize}



\section{Methodology}
* How did you collect or generate the data?
* How did you analyze the data?

Research

Testing

Learning new technology 

byggd noe kult

decribing the matherials
hvow tr

During our studies we have researched and learned the new technologies used in the Mesh potato. We have also conducted background research and looked similar work conducted before. 

Tested and further developed the descriptions provided by Village Telco. This was a big part of our assignment. a lot of the descriptions that exist on the Intranet are outdated and often hard to understand. Many of them are also not valid for the second version of the Mesh Potato.
A lot of time have gone to organize and structure information.

We created a prototype of the emergency box. While creating prototype we looked at what others have done before and their suggestions to improvements. We created a prototype and tested it, this lead to numerous suggestions for improvements. While creating something from scratch we used the method of iterations. First we planned and drew how we wanted to create the box, then developed the prototype, and conducted some tests, and made some of the improvements.  here we created the product design and how everything was put together based on similar work earlier conducted. 


Proof of value
Proof of concept

research -> insights -> Ideas --> make --> feedback


\section{Limitations}
Our main limitation was the amount of time we had available to finish our masters thesis, we only had 21 weeks at our disposal. When entering a new field it takes some time to understand the technology used. None of us have much experience with the different technologies used, and it took us some time to learn. Another limitation is money. We tried to get funding, from Engineers without Borders, to visit an area recently affected by a natural disaster. Unfortunately they had no funding available at the moment.


