\chapter{Introduction}
\label{chp:introduction} 

\section{Motivation}
Village Telco is an organization that aims to provide affordable communication in forms of data and voice services where no other companies can, or are willing to do so. Village Telco provides a “plug-and-play” solution with low cost voice and data service. While designed for the developing world, Village Telco’s solution can be applied anywhere where people wish to take control of their own telephone infrastructure.

This solution is delivered using an inexpensive fixed mesh WiFi delivery system called the Mesh Potato. The Mesh Potato unit is based on the open-source operating system, OpenWRT. Open Source telephony software combined with the latest wireless networking technology creates the potential for people to operate their own community phone systems. Mesh Potato networks have no dependence on existing telecom infrastructure, and can relatively easily be deployed anywhere in the world. It can either be deployed as a stand-alone solution or as an extension to existing technologies. Village Telco’s solution has been deployed in several countries around the world: from East-Timor and Nepal in Asia to several African and South America countries. The installed bases vary from 10 to several hundreds of Mesh Potatoes. 

As of now the Mesh Potato has mainly been permanently deployed in small villages where the existing telecommunication systems are limited, non existent or too expensive. There are many scenarios where there is need for a solution that easily and fast can provide people with telephone communication and Internet, both within a community, and with the outside world. These scenarios span from natural disasters, post-conflict situations, temporary refugee camps and IDP camps, to the use at festivals, when a mobile tower is non-functioning, or during a blackout. 

Our idea is to make an “emergency box” that consist of a Mesh Potato, a telephone, rechargeable battery, on/off switch and a solar panel to charge the battery. All this will be contained inside a robust and waterproof suitcase. All packed together and ready to go in any situation, at any time, anywhere in the world. 

Our main focus is to provide the people with Internet access, it is crucial to have the possibility to communicate with the outside world during an emergency situation. In order to get Internet into the mesh network formed by the Mesh Potatoes, at least one of the Mesh Potatoes must be connected to an access network. Which type of access network that is available depends on the location. Some places there might exist stable landlines, other places not. Then an option could be to use satellite or cellular networks. 

We hope to expand the potential of the Mesh Potato through our portable solution. We want to make it quick and easy to deploy, thus making it more useable in emergency situations. This does not only benefit the locals, but also makes the job easier for relief organizations. 
We want to provide communication where there are none,  and believe that with the “emergency box” time would be spared and lives can be saved.

\section{Problem Description}
Village Telco is an organization that aims to provide affordable communication in forms of data and voice services where no other companies can, or are willing to do so. Village Telco provides a “plug-and-play” solution with low cost voice and data service. While designed for the developing world, Village Telco’s solution can be applied anywhere where people wish to take control of their own telephone infrastructure.

This solution is delivered using an inexpensive fixed mesh WiFi delivery system called the Mesh Potato. The MeshPotato unit is based on the open-source operating system, OpenWRT. Open Source telephony software combined with the latest wireless networking technology creates the potential for people to operate their own community phone systems. Mesh Potato networks have no dependence on existing telecom infrastructure, and can relatively easily be deployed anywhere in the world. It can either be deployed as a stand-alone solution or as an extension to existing technologies. Village Telco’s solution has been deployed in several countries around the world: from East-Timor and Nepal in Asia to several African and South America countries. The installed bases vary from 10 to several hundreds of Mesh Potatoes. 

An area that has not been fully explored is the use of Mesh Potatoes in emergency situations, like natural disasters, post-conflict situations, etc. Another area to be considered is the use of Mesh Potatoes in refugee camps, where many people quickly gather in a new location. In both situations, the need for communication is essential. Key factors of usage are quick roll-out and usability. Easy to use communication is extremely important in crisis situations, both communication within the camp and outgoing communication with the rest of the world. It is important that all affected have easy access to helpful information, as this could mean the difference between life and death in some situations. In refugee camps with thousands of people, registering and reuniting people can be a difficult task to solve. Communication technology, like the Mesh Potato, could be revolutionary in situations like these. 

We will look into how communication is handled by Norwegian emergency relief organizations today, what tools they are using, and if their way of communication could be improved with the Mesh Potato. In addition to this, we will look into other existing tools, and explore the possibilities to combine them with the Mesh Potato for a better product. 

\textbf{Research questions}
- How can the Mesh Potato be used in an emergency situation like a natural disaster?

- How can an emergency box be developed?

- How is the situation, and what are the needs when it comes to communication after a natural disaster?



\section{Methodology}

\section{Limitations}
- Time
- 