\chapter{Introduction}
\label{chp:introduction} 

\section{Motivation}
As of now the Mesh Potato has mainly been permanently deployed in villages where the existing telecommunication systems are limited, non-existent or too expensive. In many scenarios there is a need for a solution that easily, and fast, can provide people with telephone communication and Internet. It may be necessary to communicate both within a community and with the outside world. The use of Mesh Potatoes as a mobile solution has not yet been fully explored. There are many scenarios where there would be useful to have a mobile communication solution. These scenarios span from natural disasters, post-conflict situations and temporary refugee camps, to the use at festivals, when a mobile tower is non-functioning or during blackouts. Communication technology, like the Mesh Potato, could be revolutionary in situations like these. 

With our portable solution, we hope to expand the Mesh Potato's potential. We want to create a solution that is quick and easy to deploy, thus making it more usable in emergency situations. This does not only benefit the locals, but also makes the job easier for relief organizations. We want to provide communication where there currently are none or the existing ones are not functioning. We believe that with the \gls{quick} box time would be spared and lives can be saved. 

Easy to use communication is extremely important in crisis situations, both communication within a community and outgoing communication with the rest of the world. Today's society rely on technology to efficiently spread information. A mobile communication system therefore enables the possibility for people to quickly spread useful information, when no other communication system is existent. 


\section{Problem Description}
As our main problem description shows, the initial approach on our thesis was to look into refugee camps and how the Mesh Potatoes could be utilized in these situations. We started contacting different Norwegian relief organizations, but found it hard to establish a good connection with any of them. We also saw that the field was enormous and to much for us to grasp with the limited amount of time we had at disposal. A deciding factor was also that we saw the need to visit a camp in order to understand how the refugee camps works, and what the need in forms of communication would be. Everyone we were in contact with expressed that no two camps are the same or run in the same way. Often camps are run by the local government, with help from the different relief organizations. Different countries have different laws and regulations, and these also have to be taken into consideration. Without being able to establish a cooperation with a relief organization early in the process, we decided to direct our focus in a slightly different direction. We therefore chose to look into the use of the Mesh Potato in different scenarios, with focus on quick roll-out and providing Internet.

Our main focus is to provide the people with Internet access, since it is crucial to communicate with the outside world during an emergency situation. In order to get Internet into the mesh network formed by the Mesh Potatoes, at least one of the Mesh Potatoes must be connected to Internet. Which type of access network that is available depends on the applicable location. Some places there exist stable landlines, other places not. Other options could then be to use satellite or cellular networks to provide the network with Internet.  

Our idea is to make an \gls{quick} box that consist of a Mesh Potato, rechargeable battery, charging regulator and a solar panel to charge the battery. All this will be contained inside a robust and waterproof suitcase. All packed together and ready to go in any situation, at any time, anywhere in the world. 

Based on our motivation we researched and conducted a study in order to answer the following research questions: 

\begin{enumerate}
\item How are the Mesh Potatoes set up? 
\item How can a \gls{quick} box be developed? What components, set-ups and configurations are necessary?
\item What kind of up-links can we connect the Mesh Potato to? And how can this be done easily? 
\item How make the roll-out process as quick as possible? What measures can we do in advance to make it as easy and fast as possible to connect the \gls{quick} box to an up-link providing Internet connection?
\item In what kind of situations could there be a need for a portable \gls{quick} box?
\end{enumerate}


\section{Methodology}

Our studies have mainly consisted of researching, and looking into the technologies used by the Mesh Potato. Before we could start to answer the questions in our problem description, it was important to get knowledge, and an understanding, of the company, Village Telco, how it all started as well as their vision. We conducted informal Skype interviews with several of the founders of Village Telco, which gave us a good insight to how it all started, how Village Telcos are created and what their motivating factors are. When conducting informal conversational interviews, there are non, or few, predefined questions in order to keep the conversation is open and adaptable as possible. Conversational interviews are a good way to establish a personal connection as well as rapidly gather information \cite{interview}. 

\begin{figure}[b]
  \centering
      \includegraphics[width=0.6\textwidth]{metode.jpg}
  \caption [Methodology for making of the \gls{quick} box]{\textbf{Methodology for making of the \gls{quick} box.}}
  \label{fig:metode}
\end{figure}


In addition to having knowledge about the company and their vision, it is important to get an understanding of the technologies utilized. This enables the possibility to conduct further research and testing. After this theoretical learning process, we started looking into how Internet can be provided to mesh networks. 

The theoretical insight gave us ideas on how to expand the Mesh Potato's area of usage. We looked at specific scenarios in need of a communication system, and this brought forward the idea of creating a \gls{quick} box that could be applied for roll-out in different scenarios. The idea of the box is developed based on previous work conducted by others. This work provided the foundation for our idea and development of the box. The prototype is provided with manuals describing how to connect a \gls{mp} to different up-links. These manuals were tested on technical and non-technical people. The testing provided us with valuable feedback, which lead to improvement of the manuals. 

We also further developed the miscellaneous set-up descriptions provided by Village Telco. This became a big part of our assignment. Many of the existing descriptions are outdated and hard to understand, and also not valid for the second version of the Mesh Potato.



\section{Limitations}
Our main limitation was the amount of time we had available to finish our masters thesis, we only had 21 weeks at our disposal. When entering a new field it takes some time to understand the technology used. None of us have much experience with the different technologies used, and it took us some time to learn. Another limitation is money. We tried to get funding, from Engineers Without Borders, to visit an area recently affected by a natural disaster. Unfortunately they had no funding available at the moment. This limited our research to be more theoretical an less practical. We were therefore not able to test our solution and manuals on the ones actually ending up using them. Instead we have conducted testing on people in our local community. These were both with technical and non-technical background, all from the younger generation. For further development the solution should be tested in a more realistic setting, as well as on a higher diversity of people. 


