\chapter{Conclusion}
\label{chp:conclusion} 

The Internet is regarded as being a human right. However, more than two thirds of the world's population are without Internet access. With the extreme weather, and climate change, natural disasters are becoming increasingly common. Whenever these disasters occur, existing communications systems often break down, or can not be utilized due to power outages. The above conditions lead to the need for a mobile communications system. We have created a solution which can be deployed all over the world and give voice services and Internet access to a larger area. This solution can be utilized by everyone, from small and large relief organizations to the man in the street. An important factor is that the system has to be affordable and easy to use. Our solutions builds on Keith Williamson's concept, the "go box". We have taken his solution to the next level, utilizing the second generation Mesh Potato, as well as including everything necessary for a quick set-up anywhere in the world, and in any situation. Hence, we named our solution the \gls{quick} box. 

QUICK stands for Quick User friendly Internet-providing Communications Kit. The Q stands for Quick, because it is easy to set up. We wrote manuals for all the different set-ups necessary to get started with the box and to provide Internet access to the network. We have also automated one set-up process by making a script. The U stands for User-friendly. The manuals we have made have been tested on both technical and "non-technical" persons, in order to make them easy to understand and as user friendly as possible. I stands for Internet-providing. The focus during our research has been on providing Internet access to the network of Mesh Potatoes. The manuals we have written guide the user through how to get Internet access to the MP by using different types of uplinks, depending on the uplink available. C stands for Communications. In today's society, and especially during emergency situations, it is crucial to have the opportunity to communicate, both within a community, and with the outside world. K stands for Kit, since this solution is delivered in a mobile suitcase ready to go in any situation. The \gls{quick} box includes a pre-configured Mesh Potato version 2, a battery to provide the MP with power, a solar panel to charge the battery, a charge regulator, necessary cables, a CD with Linux Ubuntu, a USB-stick containing the script and the different set-up manuals. 
This results in a kit that can act as a stand-alone solution. This solution can be utilized within the context of different scenarios, covering everything from emergency situations and natural disasters, to festivals and temporary refugee camps.

In the process of setting up the Mesh Potatoes, various configurations and installations were conducted. Since Village Telco is a company based on voluntary work, the descriptions found on the wiki have been partially added along the way, resulting in an extensive, but not user friendly and a rather confusing page. We found these instructions difficult to use, and spent a lot of time interpreting them. A big part of our work has therefore been to simplify these instructions, and to include them in our solution. We want the descriptions to be universal, meaning that everyone, including "non-technical" people, can run through them. These descriptions have therefore been "dumbed down", and simplified. We looked at different types of uplinks, and created manuals for connecting the MP to each of them. 

Ever since Village Telco was founded in 2008, both Village Telco's devices (MPs) and mesh networking in general have been under constant development. It is clear that Village Telco has a burning passion for providing affordable communications in rural areas. A considerable amount of new research has been conducted in this field during a short period of time. There is no doubt that mesh networking is an up-and-coming technology. This is supported by Apple's introduction of the Connectivity framework. Without hesitation, we can conclude that mesh networking is a technology we will see more and more of in the near future.  

\section{Future Work}
The research presented in this paper has only touched the surface of the Mesh Potato's potential. There are a great many aspects that need to be taken into consideration for further work in this area. The results we have presented in this paper can lay the basis for improvements of the \gls{quick} box. One possibility is to conduct a more detailed and extensive testing of the \gls{quick} box, including testing in real-world scenarios. One possibility for future work could be to make automated scripts for connecting to each uplink type. Each uplink type could have its own USB-stick. When Internet connection is desirable, the user could then simply plug in the corresponding USB-stick, and the rest of the set-up would proceed automatically. This emphasizes Village Telco's vision of a plug-and-play solution. Another future possibility for the Mesh Potato, is to enable the possibility to communicate with other commercial mesh networks, such as Apple's. The possibilities are virtually infinite, since this is a complex and a very hot topic.

