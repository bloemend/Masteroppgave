

\begin{enumerate}
\item Connect the MP (make sure the MP is powered up) to the PC, running Linux Ubuntu, with an Ethernet cable. The Ethernet cable must be put into the LAN-port on the MP. 

\item Open Linux terminal by pressing "Ctrl+Alt+t" and install telnet, dns and iptables by entering the following commands: 
\noindent
\begin{lstlisting}[language=bash]
 $ sudo su
 $ apt-get install telnetd
 $ /etc/init.d/openbsd-inetd restart 
 $ apt-get install dnsmasq
 $ apt-get install iptables
\end{lstlisting}

\item The Mesh Potato will be pre-configured with the IP address 192.168.1.x. In order to access the MP, the PC must be on the same subnet. If the PC has only one Ethernet port, this is most likely called eth0. If the PC has two or more Ethernet ports, make sure to change from eth0 to the correct port-name in the following command. Execute the following command to enter the PC into the same subnet as the MP: 
\noindent
\begin{lstlisting}[language=bash]
  $ ifconfig eth0 up 192.168.1.2
\end{lstlisting}

\item Open a browser on your PC and type in "192.168.1.x" in the URL field. Remember that x is equal to the last number of the IP address stated on the MP. The SECN Web Interface should now appear. This assures you that you have contact with the Mesh Potato. Changes in the interface will be described further down, so do not close this window.  

\item  
If there is only one Ethernet port on the PC:
\begin{itemize} 
\item [] \textbf{<WIRELESS INTERFACE> = wlan0}
\item [] \textbf{<INTERFACE CONNECTED TO MP> = eth0}
\end{itemize}

(This is not common, since most laptops only have one Ethernet port, but if there are two or more Ethernet ports on the PC, make sure to use the right ports in the following commands. In order to find out which ports are in use, execute the command "ifconfig" in the terminal. The ports in use will have a "inet addr" stated.) 

Go to the terminal and write the following commands in order to set up the ip tables correctly:
\noindent
\begin{lstlisting}[language=bash]
  $ iptables --table nat --append POSTROUTING --out-interface
   <WIRELESS INTERFACE> -j MASQUERADE
  $ iptables --append FORWARD --in-interface <INTERFACE 
  CONNECTED TO MP> -j ACCEPT
  $ echo 1 > /proc/sys/net/ipv4/ip_forward
\end{lstlisting} 
\begin{itemize}
\item If you mess up in this step, accidentally write something wrong etc., the following commands will reset the ip tables, and you may try step 5 again.
\noindent
\begin{lstlisting}[language=bash]
  $ iptables --table nat --flush
  $ iptables --flush
  $ iptables --delete-chain
\end{lstlisting}
\end{itemize}  

\item Execute the following commands to telnet into the \gls{mp} and configure the default gateway: (Note that the output of the command "route del default" may display "SIOCDELRT: No such process". This is because the MP does not have a default gateway. Ignore this notification.) Remember that x is equal to the last number of the IP address stated on the MP.
\begin{lstlisting}[language=bash]
  $ telnet 192.168.1.x
  $ route 
  $ route del default 
  $ route add default gateway 192.168.1.2
\end{lstlisting} 

We highly recommend that you enhance the security on the \gls{mp} by enabling SSH instead of telnet. This has no impact on this set-up, and will not affect the ability to provide Internet to the network. Description on how to enable SSH is found in section \ref{subsubsec:ssh}.

\item Go back to the web interface and click on the "Advanced"-tab at the top of the page. Change the following parameters under "DHCP Server":
\begin{itemize}
\item Tick the box "Enable DHCP Server".
\item Remove the tick from "Use device IP".
\item Change the address in "Gateway Router" to "192.168.1.2".
\item Press "Save" at the bottom of the page. 
\end{itemize}

\item Internet should now be available in the mesh network. You can test this by connecting to the \gls{mp} from, for example, your smart phone or a PC. 
\begin{itemize}
\item The name of the network: \textbf{MP2_21}
\item The password is: \textbf{potato-potato}
\end{itemize}
Both the name and the password can be changed in the web interface. 
\end{enumerate}
