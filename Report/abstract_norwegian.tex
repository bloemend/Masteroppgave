\renewcommand{\abstractname}{Abstract}
\begin{abstract}


* Hvorfor vi gjør det som vi gjør, hva er bakgrunnen

* Hva har vi gjort

* hvordan har vi gjort det

* hva fant vi ut

* hva resulterer det i hva konkluderer vi med?



Village Telco is an organization that aims to provide affordable communication in forms of data and voice services where no other companies can, or are willing to do so. Village Telco provides a “plug-and-play” solution with low cost voice and data service. While designed for the developing world, Village Telco’s solution can be applied anywhere people wish to take control of their own communication infrastructure.

This solution is delivered using an inexpensive fixed mesh WiFi delivery system called the Mesh Potato. The Mesh Potato unit is based on the open-source operating system, OpenWRT. Open Source telephony software combined with the latest wireless networking technology creates the potential for people to operate their own community phone systems. Mesh Potato networks have no dependence on existing telecom infrastructure, and can relatively easily be deployed anywhere in the world. It can either be deployed as a stand-alone solution or as an extension to existing technologies. Village Telco’s solution has been deployed in several countries around the world: from East-Timor and Nepal in Asia to several African and South America countries. The installed bases vary from 10 to several hundreds of Mesh Potatoes. 

We have directed our studies toward the use of Mesh Potatoes in mobile situations. We started by looking into different scenarios consisting of everything from emergency situations and natural disasters, to festivals and temporary refugee camps. After being in contact with different relief organization we got a clear impression that communication under and after a disaster situation is a difficult task. Keith Williamson, a Village Telco volunteer, has created a "go box" using the first generation of the Mesh Potato. We wanted to take his solution further and develop a mobile and stand-alone kit that could be used in all the different scenarios that we have taken into consideration. The prototype we developed differs from the one already made. We have used the second generation Mesh Potato and the box includes everything necessary, solar panel, battery, different cables and easy to use manuals for configuration. 

In the process of creating this kit we had to set up the network in order to conduct testing, both of the network and the configurations. We found that many of the descriptions found on the Village Telco wiki were little explanatory and difficult to use. Therefore a big part of our work has been to simplify these descriptions, and make them easy that little or non-technical people can use and understand them. We have looked at simplifications as well as processes in order to make the roll-out as quick as possible. Hence we have named our solution the \gls{quick} box.

* Fylle ut mer her etter at vi har skrevet konklusjonen

We believe that our work and theoretical research has contributed and richened Village Telco. And eventually save lives. 






\end{abstract}