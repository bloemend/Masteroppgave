\chapter{Main Work}
\label{chp:quickrollout} 

Make a best practice for quick roll out, by the use of a MultiBox. 

\section{Set-up of the Mesh Potatoes}
%Flashing process
%How did we get Internet to the network?

\section{The Emergency Box}

\begin{center}
\begin{table}[!ht]
\caption{\label{tab:components}The components of MultiBox}
    \begin{tabular}{ | l | p{9cm} |}
    \hline
    \textbf{Component} & \textbf{Description and purpose} \\ 
    \hline
    Mesh Potato &  \\ 
    \hline
    Suitcase/box &   \\ 
    \hline
    Power supply & \\
    \hline
    Plain old telephone & \\
	\hline
	Junction box & \\
	\hline
	 & \\
	\hline
    \end{tabular}
   \end{table}
\end{center}

%How is the box set up?
%How is it made?
%How does it work?
%Explain the different components
%Explain evt. changes
%Script?

\section{Up-Link}
Our main focus when deploying the emergency boxes, is to provide Internet to the mesh network. This is because it is crucial to have the possibility to communicate with the local community and the outside world during an emergency situation. In 2011, UN declared Internet access a Human Right \cite{HR}. This says something about the extent of the Internet, and the importance of connectivity. In order to provide Internet to the mesh network formed by the emergency boxes, at least on the the Mesh Potatoes must be connected to an access network via an uplink. An uplink connects a device or a LAN to a larger network \cite{uplink}. Which type of access network that is available depends of the location. Some places there might exist a stable landline, other places not. Then an option could be to use satellite or cellular networks. It is therefore important that the emergency box has high adaptability in order to fit different scenarios. In the following sections, we will look at some of the uplinks available, and how Internet access can be provided to the mesh network.  

%I hvilke tilfeller de passer best? 
%Hva slags utstyr for hver up-link må til??

\subsection{Landline}
The most common way of getting Internet access is via a landline. This is usually in the form of twisted pairs (copper lines). These lines support broadband up to 10 Mbps, and are often in form of ADSL, or other digital subscriber line of type x (xDSL) technologies \citep{audestad}. 

\subsubsection{Telephone Line}
\subsubsection{Internet Service Provider}

\subsection{Mobile Network}
\subsubsection{UMTS - 3G}
\subsubsection{4G}

\subsection{Satellite}
%ulovlig i feks india, hva skjer da? http://en.wikipedia.org/wiki/Satellite_phone

Internet from satellites are offered by a satellite Internet provider \cite{cablevssatellite}. The satellite are orbiting the Earth, and get signals from a land based Internet connection. To get Internet access via satellite you need a satellite dish. The main advantage of using satellite is that it provides an universally available Internet access \cite{broadband}. Since it is universally available, it is fitted for use in rural regions where there exists no landlines or other options for connecting to the Internet. But like with everything else, there also exists disadvantages with using satellite-Internet. Since it is a shared medium, privacy concerns arise, and the speed are dependent of simultaneous use. Another disadvantage is that the connection can be affected by bad weather, unlike for a wired connection.

\subsection{Apple's Mesh Network}
In Apples iOS 7 a new framework have become available; a multi-peer connectivity framework \cite{appleMesh}. 



The Multipeer Connectivity framework provides support for discovering services provided by nearby iOS devices using peer-to-peer WiFi, infrastructure Wi-Fi networks and bluetooth to communicate with those services. This communication could either be message-based data, streaming data and resources such as files \cite{multipeer}.

\subsection{Internet Balloons}


\section{Different Scenarios Where a Quick Roll-out is Necessary}
%What are the specific need for the different scenarios? What adjustments are necessary?

Everyday there are situations all over the world that in some way affects the modern communications systems, or causes a need for one. These situations can range in everything from big natural disasters like the tsunami in Japan or the volcano outbreak in the Philippines, where either parts of the communication system is not functioning or there is a desperate need for one. To temporary refugee camps and IDP camps, and situations where a mobile tower is down, or blackouts. Also more festive situations can have use of the quick roll-out system. Imagine a big group gathered at a festival in another country. It is expensive to call or use mobile data, to be able to use cheap internet via the mesh Potatoes would save the users for a lot of money. 

\subsection{Natural Disasters}
%Asia og Sør-Amerika
%Philippines - contact Kenneth Bjerkelund to find out their needs for internet etc. makingchange.no. 
A Natural disaster id defines as; \textit{any event or force of nature that has catastrophic consequences, such as avalanche, earthquake, flood, forest fire, hurricane, lightning, tornado, tsunami, and volcanic eruption} \cite{naturalDisaster}.


\subsection{Blackouts}

\subsection{Temporary Refugee and IDP camps}
Not all refugee and IDP camps are as well established, like the ones in Dadaab. Many camps are temporary, and are therefore in more need of a temporary communication system. 

\subsection{Breakdown of Mobile Towers}


The 10th of June 2011 one of Telenor had problems with one of its servers in Oslo. This problem caused a down time of 18 hours and affected 3 000 000 Telenor users \cite{listeNedetid}. Not only was this the biggest problem Telenor have had since they opened their mobile network in 1993, but also the longest downtime and highest number of affected users recorded in Norway. In addition to this it all happened in a period with severe flooding in big parts of the eastern Norway, and made it difficult to reach emergency numbers. The fact that the problems occurred during the flooding just made the situation much worse. \cite{TelenorNede}

\subsection{Mountain Areas - Avalanches}

\subsection{Festivals}

\subsection{Differences/relation between different scenarios}
%What kind of adjustments are needed between the different scenarios?


%Mopedtilfellet
%Raspberry Pi


\section{The Process of Quick Roll-out}

\subsection{How are telephone numbers assigned?}
%How do people know what number to call?

\subsection{Training of People}