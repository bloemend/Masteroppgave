\chapter{Main Work}
\label{chp:quickrollout} 

Make a best practice for quick roll out, by the use of a MultiBox. 

*Alt som står på norsk er kommentarer på hva mer som skal være med.

*Her skal det være en mer utfyllende innledning til kapittelet. Hva kap. inneholder.


\section{Set-up of the Mesh Potatoes}
The set-up of the Mesh Potato is a process including installing firmware, allocating IP-addresses, providing Internet to the network and so on. The firmware used for the Mesh Potatoes is called Small Enterprise/Campus Network (SECN) Firmware \cite{ChoosingFirmware}. A flashing process is applied in order to install/update the firmware on the Mesh Potato. See the SECN User Guide in Appendix \ref{chp:appendixB} for a more detailed description of how to set up the Mesh Potatoes and how different networks can be built. 

\subsection{Installing Firmware on Mesh Potato 1.0}

Flashing is the process of updating or changing the firmware (SECN) on the Mesh Potato. We updated the firmware by using the potato-flash application \cite{flashing}. This is a specialised software application for the Mesh Potato. Potato-flash can be used regardless of previously installed firmware on the Mesh Potato \cite{InstallingSecnFirmware}. 

\begin{enumerate}
\item Downloaded the 64 bit potato-flash utility from \url{http://download.villagetelco.org/utilities/potato-flash/potato-flash-64bit/} to the folder \textbf{/etc/local/bin}.
\item Made the potato-flash file executable by writing the following command in the Linux terminal.
\begin{lstlisting}[language=bash]
  $ cmod +x /usr/local/bin/potato-flash-x64
\end{lstlisting}
\item Downloaded the rootfs file (\texttt{openwrt-secn1_1-GA01-MP01-root.suashfs}) and the kernel file (\texttt{openwrt-secn1_1-GA01-MP01-vmlinux.lzma})  from \url{http://download.villagetelco.org/firmware/secn/stable/mp/SECN-1.1/} to a folder we called mp\_firmware on our local directory.
\item Opened the terminal and wrote the following commands: 
\end{enumerate}


\begin{framed}
\noindent \textbf{Enter root environment:} 
\begin{lstlisting}[language=bash]
  $ sudo su
\end{lstlisting}

\noindent \textbf{Turn of network manager:}
\begin{lstlisting}[language=bash]
  $ service network-manager stop
\end{lstlisting}

\noindent \textbf{Bring the interface connected to the MP up:}
\begin{lstlisting}[language=bash]
  $ ip link set eth0 up
\end{lstlisting}

\noindent \textbf{Go into the directory containing the .squashfs and .lzma files:}
\begin{lstlisting}[language=bash]
  $ cd <the directory containting the .squashfs and 
  .lzma files>
\end{lstlisting}

\noindent \textbf{Assign IP-address to the interface:}
\begin{lstlisting}[language=bash]
  $ ifconfig eth0 1.1.1.1
\end{lstlisting}

Before running the potato-flash utility we made sure that the power was unplugged, and that the Mesh Potato was connected to our PC via an Ethernet cable. 

\noindent \textbf{Executing the potato-flash utility:}
\begin{lstlisting}[language=bash]
  $ potato-flash-x64 openwrt-secn1_1-GA01-MP01-root.suashfs 
   openwrt-secn1_1-GA01-MP01-vmlinux.lzma
\end{lstlisting}
\end{framed}

Briefly after the potato-flash had been executed, dots stared appearing on the screen, like shown in \fref{fig:flashing}. When these dots started appearing, we plugged the power supply back into the Mesh Potato, and the process of upgrading the firmware started. 

\begin{figure}[t]
  \centering
      \includegraphics[width=1\textwidth]{potatoFlash.png}
  \caption [Flashing the Mesh Potato]{\textbf{Flashing the Mesh Potato.} This figure shows the flashing process from when we first flashed our Mesh Potato 1.0}
  \label{fig:flashing}
\end{figure}


\subsection{Installing Firmware on Mesh Potato 2.0}
*Her skal vi skrive om flashing prosessen, hvordan få internett til nettverket, hvordan  en MP kan koble seg til de ulike uplinkene, 3G via TP-links til mesh nettverket (?). 
%Flashing process
%How did we get Internet to the network?
%Hvordan en MP kan koble seg til de ulike uplinkene?
%3G via TP-links til mesh nettverket?

\subsection{Example Mesh network}
We first started experimenting with the Mesh Potatoes by setting up a small network consisting of two MPs each connected to a plain old telephone, and a computer running Ubuntu Linux. This simple network is shown in \fref{fig:network}. Each of the MPs were assigned a static IP address, these addresses were not part of the LAN address space. The IP addresses are allocated in a predefined default address space 10.130.1.xxx. To administrate the MP devices we used a workstation linked together with one of the MPs in the network by using a Ethernet cable (can also be connected via Wi-Fi). We assigned the workstation a static address within the same address space as the MP devices. We were able to make phone calls between the Mesh Potatoes by dialling the last octet or the whole IP address.

\begin{figure}[t]
  \centering
      \includegraphics[width=1\textwidth]{network1}
  \caption [Example of a simple mesh network]{\textbf{Simple mesh network.} This figure illustrates a simple mesh network with the use of Mesh Potatoes.}
  \label{fig:network}
\end{figure}

\subsection{The SECN Web Interface}
Village Telco provides a SECN web interface for configuration and management of individual MPs. This web interface can be accessed by entering the IP address of the MP in a browser, this interface is shown in \fref{fig:webinterface} In order to be able to do this, the PC must be on the same subnet (exact same prefix) as the MP.  

\begin{figure}[t]
  \centering
      \includegraphics[width=1\textwidth]{webinterface.png}
  \caption [Web interface]{\textbf{Web interface} Displays the web interface to the Mesh Potato.}
  \label{fig:webinterface}
\end{figure}



\section{The Emergency Box}
The Mesh potatoes and Village Telcos were created to get voice and data connection to areas where they are non existent or too expensive for the average person. As of now the Mesh Potatoes have been set up to create small or bigger networks in villages in big parts of the world. We want to expand the solutions, and look at its mobility, its usage on the go. We want to make a box that has high adaptability, enabling it to easily be used in several different scenarios, under different conditions and by different people. The box have to be so easy that there are no room for misunderstanding or mistakes. It should be a ready-to-go box ready for any situations.  
 
\subsubsection{Previous/similar work}
Something similar has been done before by Keith Williamson, a volunteer in the Village Telco community. His idea of utilizing the Mesh Potatoes in emergency situations started with his interest in amateur radios, and the use of radios in emergency situations. He put together a "go box" by using a waterproof Pelican 1200 case. This case contained a bracket holding the Mesh Potato, a rechargeable Li-Ion or Li-Poly battery, telephone handset, and to connect it all together and get a on/off switch, a junction box was included \cite{keith}. 

*Se om vi finner minst et eksempel til på noe lignende som er gjort før. 

*Hvordan er vår boks annerledes fra Keith sin, hvordan er den satt opp, hvordan er den laget, hvordan fungerer den, forklare ulike komponentene, forklare endringer som gjøres underveis grunnet testing (flere versjoner), ting som gjøres for at den kan rulles ut raskt (feks script). 
%Hvordan er den annerledes fra Keith sin?
%How is the box set up?
%How is it made?
%How does it work?
%Explain the different components
%Explain evt. changes
%Script?

\subsection{The Different Components}
The components of our emergency box is described in \tref{tab:components}. 

\begin{center}
\begin{table}[t]
\caption{\label{tab:components}The components of MultiBox}
    \begin{tabular}{ | l | p{9cm} |}
    \hline
    \textbf{Component} & \textbf{Description and purpose} \\ 
    \hline
    Mesh Potato &  \\ 
    \hline
    Suitcase/box &  A suitcase made of high quality plastic coated with aluminium foil. Strengthened edges and corners of aluminium and steel. It has a soft-padded interior, than can be split into different rooms. Two snap locks with keys, and a solid handle for carrying. Dimensions: 455x330x152 (width, depth, height). Weight: 2,6 kg. \\ 
    \hline
    Power supply & A gel battery (12 V and 5 Ah). No need for maintenance. The battery acid is bound in a viscous gel. This prevents leakage, even when the battery is mounted horizontally. Long lifetime and safe to handle. The battery is fully closed, and do not need refill of battery water. No hydrogen gas or other gas might leak. When the battery is charging no gas or acid vapor is emitted, hence the battery can be placed in narrow or enclosed spaces. The battery withstands multiple discharges. The gel battery is ideal for seasonal or occasional use, since it have a slow self-discharge tempo, and a good ability to recover after deep discharging. Dimensions: 114x69x109. Weight: 2,16 kg. \\
    \hline
    Plain old telephone & \\
	\hline
	Junction box & \\
	\hline
	Solar panel & Solar panel from Multicomp with item number: MC-SP10-GCS. Power rating: 10 W. Power Voltage Max: 17 V. Dimensions: 357x280x18\\
	\hline
	Charging regulator & Regulator for 12 V solar panel. Protects the battery from overcharging and discharge. The charging regulator is connected between the solar panel and the battery to regulate the voltage to the battery. Capacity: 100 W / max. 7 A. Overcharging protection: 14.5 V. Discharge protection: <10,5 V. Three diodes shows charging, high voltage and low battery voltage. \\
	\hline
    \end{tabular}
   \end{table}
\end{center}

\subsection{Creating the Multibox}
The main part that binds all the components together is the solar regulator. This is used in order to not overcharge the battery. The solar regulator was initially built for an other type of solar panel tan the one we chose to use. This, for example, means that the concomitant plugs could not be used. These plugs were cut of and we soldered on our own wires. The solar regulator is connected to all three main components, the solar panel, the battery and the Mesh Potato. 

First we cut of the plugs on the solar regulator and soldered on new wires. Then we cut of part of the power cord to the MP and soldered onto one of the wires. The second wire from the solar regulator was connected to solar panel. Before doing this we measures how much voltage was in the solar panel and it this voltage was affected by light. This test was done just by holding the solar panel up to a bright lamp. When both the solar panel and the MP was connected to the regulator, the only thing remaining was the battery. The battery was fully charged and showed a voltage of a little over 13 V. We unpligged the MP and connected the battery. After the battery was attached we powered up the MP, and it worked. 

This is a very easy setup, and have room for many improvements. As a strated the most important aspect is that is works. 


\paragraph{Possible improvements }
As mentions this setup is fairly easy and simple, but leaves room for improvements. For a starter there should be a on and of switch to not use battery capacity when not in use. This switch should be on the wire between the battery and the regulator. Unless we use a tool to check the voltage there are no way for us to know how much of the battery that has been used. This could be very useful when in a situation where communication is vital and the hours of lifetime should be thought about. 
The battery we are utalizing is a gel battery, they are less explosive that the lithium batteries, and are allowed to take on planes, but on the contrary they are heavy. If the aim was to make a case that is as portable as possible a lighter battery would be necessary. 

The MP2 is a lot smaller than the fist generation, and a powerful solar panel dores not take that much space so there is no need for a suitcase of the size we have purchased. A smaller care that is lighter and easier to handle would make the solution even more portable. 



* How make it suitable to connect to different up-links, how to the configuration process as easy as possible. 

\subsection{Battery and Charging Calculations}
All the calculations done in this section is based on the equipment described in \tref{tab:components}.

\paragraph{Charging with solar panel.}
How long it will take for the solar panel to charge the battery from fully discharged to fully charged depends on how much sun there is, but with the peak value of the solar panel (10 W) it takes: 

The solar panel capacity is:
$$Amp = \frac{Watt}{Volt} = \frac{10 W}{17 V} = 0.59 Ampere$$

If the battery is completely discharged it will take: 

$$Amp\times Hours = AmpHours \Rightarrow Hours =\frac{AmpHours}{Amp} = \frac{5 Ah}{0.59 A} = 8.47 Hours$$

to fully charge it. 

\paragraph{With Mesh Potato v1.0}
With the components described in \tref{tab:components} the number of hours the MP1 can last with fully charged battery is: 

$$Amp = \frac{Watt}{Volt} = \frac{2.5 W}{12 V} = 0.208 Ampere$$
$$Amp\times Hours = AmpHours \Rightarrow Hours = \frac{AmpHours}{Amp} = \frac{5 Ah}{0.208 A} = 24 Hours$$

\paragraph{With Mesh Potato v2.0}
With the components described in \tref{tab:components} the number of hours the MP2 can last with fully charged battery is: 

$$Amp = \frac{Watt}{Volt} = \frac{? W}{12 V} = ? Ampere$$
$$Amp\times Hours = AmpHours \Rightarrow Hours = \frac{AmpHours}{Amp} = \frac{? Ah}{? A} = ? Hours$$

\subsection{The Process of Quick Roll-Out}
*Forklare spesifikke ting som er tatt i betrakning for å gjøre utrullingen raskest mulig. Feks scripting, manual etc. 

*Hvordan utdele telefonnummer raskt, opplæring av folk, 


\section{Up-Link}
Our main focus when deploying the emergency boxes, is to provide Internet to the mesh network. This is because it is crucial to have the possibility to communicate with the local community and the outside world during an emergency situation. In 2011, UN declared Internet access a human right \cite{HR}. This says something about the extent of the Internet, and the importance of connectivity. In order to provide Internet to the mesh network formed by the emergency boxes, at least on the the Mesh Potatoes must be connected to an access network via an uplink. An uplink connects a device or a LAN to a larger network \cite{uplink}. Which type of access network that is available depends of the location. Some places there might exist a stable landline, other places not. Then an option could be to use satellite or cellular networks. It is therefore important that the emergency box has high adaptability in order to fit different scenarios. The availability of the different uplinks is not the only thing that vary. The up-link speed and the price also varies from place to place, but also varies between the types of uplinks. In the following sections, we will look at some of the uplinks available, and how Internet access can be provided to the mesh network.  

\subsection{Internet via Telephone-line}
The most common way of getting Internet access is via a landline. The telephone lines are most often used for this, since they can be converted to broadband. In this way it can be used for phone calls and Internet simultaneously \cite{internet}. The line is usually in the form of twisted pairs (copper lines). These lines support broadband up to 10 Mbps, and are often in form of ADSL, or other digital subscriber line of type x (xDSL) technologies \citep{audestad}. Internet via telephone lines can be provided as a stand-alone solution, or it can be provided together with television or/and phone service. The latter option is usually cheaper. Internet through landlines have a high reliability \cite{cablevssatellite}. We will now shorty describe some technologies for getting Internet access via a telephone line; dial-up Internet connection, ISDN, and DSL. Although dial-up Internet connection is practically extinct in developed countries, we will include it here due to the different application scenarios for the emergency box. 

\paragraph{Dial-up Internet connection}
Dial-up is an analogue technology that utilizes the telephone line. A telephone wall jack is used as a fixed point of connection, and the computer is connected to a voiceband modem. With this technology, the data is transmitted over the same frequencies used for phone calls. Hence, if you only have one telephone line, you cannot take a phone call and use Internet at the same time \cite{differentuplinks}. Along with the digital era, better internet technologies were introduced; ISDN and DSL. 

\paragraph{ISDN}
Integrated Services Digital Network (ISDN) is a fixed internet connection, which also utilizes the telephone lines. When using ISDN, as with dial-up, a telephone wall jack is used as a fixed point of connection. But ISDN utilizes a ISDN terminal adapter instead of voiceband modem. This ISDN terminal adapter sends out digital signals. The data speed varies between 64 Kbps - 129 Kbps. The speed of the data is symmetric, which means upstream and downstream data rates are the same. In contrary to dial-up, ISDN allows voice calls and transmission of data simultaneously. ISDN is faster than dial-up, but the speed is nothing compared to the speed obtained using DSL \cite{differentuplinks}. 

\paragraph{DSL}
Digital subscriber line (DSL) is, like the name indicates, a digital high-speed technology for Internet access that allows simultaneous voice and data transfer. Like dial-up and ISDN, DSL also run over the telephone lines. With DSL the data is not converted between analogue and digital signals. Despite this, the signals are modulated in order to be transferred on non-voice frequencies. DSL is an always-on technology, and differ from the previous technologies mentioned in this way. Only a small part of the telephone line is used for voice signals. The DSL technology allows utilization of a unused frequency spectrum of an telephone line, hence making it possible to transmit data faster. When the voice and data signals arrive at the telephone company's local switching station, they are separated and routed differently; voice to regular telephone system and data to the ISP, and then the Internet. A connection must be within approximately 5 kilometres of a station in order for DSL to work. The speed depends on many factors. Data can be transported up to 6 Mbps (distance of approximately 2 kilometres). Relevant factors that have an impact on the speed is distance to the switching station and the quality of the telephone line. Like mentioned earlier, there are different types of DSLs. The most common is ADSL, where the A stands for asymmetric; the downstream speed is faster than the upstream speed \cite{differentuplinks}.

\subsubsection{Quickly Connect the Emergency Box to This Uplink}
*Forklare hvordan man raskest mulig kobler nødboksen til akkurat denne type uplink.

\subsection{Cellular Network Technologies}

\begin{figure}[b]
  \centering
      \includegraphics[width=1\textwidth]{mobilesubscriptions.png}
  \caption [Number of mobile-cellular subscriptions]{\textbf{Number of mobile-cellular subscriptions} The figure shows that the growth of mobile-cellular subscriptions have increased drastically during the last decade, and show that there are almost as many as there are people in the world \cite{itu2013}.}
  \label{fig:subscribers}
\end{figure}

%ITU2011-artikkel
It is getting more and more common to use cellular technologies for broadband. Around 2011 number of mobile-broadband subscriptions grew to twice as many as the number of fixed-broadband subscriptions. In developed countries it is common to have a fixed-broadband connection, and use a mobile-broadband network in addition to the fixed. In developing countries on the other hand, it is not a given that there is access to a fixed-broadband connection. Then mobile-broadband can be the only method of access available. In 2011, 90 \% of the world's population had 2G coverage, and 45 \% had 3G coverage \cite{itu2011}. By 2013, the number of mobile-cellular subscriptions had reached a high level, and were approaching the number of people in the world, like shown in \fref{fig:subscribers}. From 2011 to 2013, the number of mobile-broadband  subscriptions more than doubled in developing countries \cite{itu2013}. 

Through mobile network technologies, high-speed Internet access can be provided via portable devices. In order to get mobile broadband, there must be a cellular network (GSM, CDMA) service available. The key technologies when talking about mobile broadband is 3G and 4G (respectively third and fourth generation wireless networks) \cite{mobilebroadband}. With 3G the average speed is 0.5 to 1.5 Mbps, and with 4G the average speed is 2 to 12 Mbps. These vary, due to different versions of each technology, underlying service etc.. And like with everything else, the actual and realistic speed differ from the peak speed \cite{3gvs4g}. 

\subsubsection{Quickly Connect the Emergency Box to This Uplink}
*Forklare hvordan man raskest mulig kobler nødboksen til akkurat denne type uplink.


\subsection{Satellite}
%ulovlig i feks india, hva skjer da? http://en.wikipedia.org/wiki/Satellite_phone
Internet from satellites are offered by a satellite Internet provider \cite{cablevssatellite}. The satellite are orbiting the Earth, and get signals from a land based Internet connection. To get Internet broadband via satellite you need a satellite dish. The main advantage of using satellite is that it provides an universally available Internet access \cite{broadband}. Since it is universally available, it is fitted for use in rural regions where there exists no landlines or other options for connecting to the Internet. But like with everything else, there also exists disadvantages with using satellite-Internet. Since it is a shared medium, privacy concerns arise, and the speed are dependent of simultaneous use. Also the connection can be affected by bad weather, unlike for a wired connection, hence it is not as reliable as cable. 

\subsubsection{Quickly Connect the Emergency Box to This Uplink}
*Forklare hvordan man raskest mulig kobler nødboksen til akkurat denne type uplink.

\subsection{Summary Up-Links}

\begin{center}
\begin{table}[!h]
\caption{\label{tab:uplinks}Advantages and disadvantages - Up-links \cite{comparisonuplinks}.}
    \begin{tabular}{ | l | p{4cm} | p{5cm} |}
    \hline
    \textbf{Up-links} & \textbf{Advantages} & \textbf{Disadvantages} \\ 
    \hline
    Landline/xDSL & High reliability, cost effective, good speed. & Low availability in rural areas. \\ 
    \hline
     Cellular networks & High availability, fitted for "on the move"-use. & Expensive, slower than xDSL.\\
    \hline
    Satellite & High availability.  & Unreliable, expensive, slower than landline.\\ 
    \hline
    \end{tabular}
   \end{table}
\end{center}

*Forklare ulikhetene og likhetene ved å koble nødboksen til de forskjellige uplinkene. 

\section{Future Internet Access Methods}
Different methods of distributing Internet is always under development. The previous up-links described is well established, but in many parts of the world not fully developed or not affordable to the average person. The large technology companies, like Google, are experimenting with different ways of distributing Internet. 

\subsection{Google's Internet Balloons}
The majority of the world today is not connected to the Internet. Two thirds of the worlds population does not have access. Project Loon, a Google project, is a network of high altitude balloons travelling in the stratosphere, and through this network be able to give Internet to the Entire world. Cost effective, reliable and inexpensive internet connection to everybody. The project started in June 2013 as a experiment in New Zealand \cite{loon}. 

\begin{figure}
        \centering
        \begin{subfigure}[t]{0.43\textwidth}
                \includegraphics[width=\textwidth]{loon2.PNG}
                \caption[The Balloons are situated in the stratosphere]{\textbf{The Balloons are situated in the stratosphere.}} 
                \label{fig:loonStratosphere}
        \end{subfigure}%
        ~ %add desired spacing between images, e. g. ~, \quad, \qquad etc.
          %(or a blank line to force the subfigure onto a new line)
        \begin{subfigure}[t]{0.415\textwidth}
                \includegraphics[width=\textwidth]{loon1.jpg}
               \caption[Connecting to the Internet]							{\textbf{Connecting to the Internet}} 
                \label{fig:loonConnect}
        \end{subfigure}
        ~ %add desired spacing between images, e. g. ~, \quad, \qquad etc.
          %(or a blank line to force the subfigure onto a new line)
        \caption{Project Loon: Balloon-powered internet for everyone.}\label{fig:loon}
\end{figure}

The balloons are 15 meter in diameter. They travel about 20 km up in the air, in the stratosphere, twice as high as airplanes and weather, this is shown in \fref{fig:loonStratosphere}. At this altitude there are many layers of wind, each varies in direction and speed. By regulating which wind the balloons are flying in it is possible to control their position, and steer them in the desired direction. \fref{fig:loonConnect} show that  people can connect to the Internet shared by the balloons by having a special antenna attached to their house. From this antenna the signal bounces to one balloon which again bounces through the other balloons and down to the Local Internet Provider on earth. This creates a network in the sky. 

In order to control the altitude of the balloon, there is a special designed control system. The system is managed remotely from the ground. By either pumping air in, or letting air out of the balloon, one can decide in what layer of air the balloon should be in. Letting air in and out is not the only way to decide is the balloon should go up or down, but it is the only way to do so in huge scale. A GPS is attached to each balloon in order to keep track of precise positions and see how the winds are changing. There are enormous amounts of data collected. The balloons are flying at the same speed as the wind.  

The balloons contains specially designed antennas and radio systems. This in order to receive signals from Project Loon only, in order to achieve high bandwidth over long distances. Satellites stay at the same place and at the same altitude. This means that the satellite dish can be mounted in the right direction. This is not the case with the balloons, they are in constant movement and they also vary in attitude. A fixed pointing dish would therefore not work. The antenna needs more sensitivity to an angle than it does straight up, which results a uniform signal strength. 

Since the balloons are in constant movement, it is important to make sure that there always is a balloon available and one ready to move in when one move out so that the connectivity always are available. If not this project would not be of much use. Every balloon contains information about all the other balloons in order to spread out nicely. Think of it as a flock of birds, they always look at the one next to them and space out nicely.

The balloons are solely run on solar power. The balloon works as a communication tower in the sky. The solar panels catches the sunlight that is available during the day, as well as charging up the lithium-ion battery to last the night. In the stratosphere it is -70 degrees Celsius, these extreme cold conditions are not ideal fro the lithium battery. In order to make sure that the battery does not loose its effective battery capacity it is important that the battery is kept warm. The battery pack is insulated to reflect the heat that comes of the electronics to stay warm. This is still under development. 

When it comes to lifetime, the goal is that the balloons stay alive for a 100 days, or 3 laps around the globe. It is important to make sure that the balloon is leak free. The material is under a lot of stress, air is constantly being pumped in and out to control the position.. As well as the extreme cold and warm temperatures. 

Both on the way up and down the air traffic control in the specific country have to be contacted since they go through airspace. Project loon requested permission to land on Norwegian soil according to Teknisk Ukeblad, a Norwegian science magazine. This permission was approved \cite{loonTU}.

The area of usage is enormous, it is everything. Situations where the Internet infrastructure has not been built out, either because it is too expensive or just not possible. Emergency situations, natural disaster and other situations where the cell towers are down and causes the need to communicate. We can also see the political aspect of it. With these balloons the Chinese people can finally have access to a free and uncensored Internet. Project Loon is still a development project with extensive testing. The project are taking into use what everybody have in common, namely the sky, to reach out to all the areas where access has not been possible. It is expensive to have enough balloons in the sky in order to have good enough coverage. Also the speed is not really high. And even though a specialized antenna is required to get access, this is breakthrough in order to get the whole world connected \cite{loonYouTube, loonNorsk}.

Although this solution is not an option for distributing Internet as of today, it could be a good option in the near future. The solution is affordable and well compatible with our emergency box solution.

Iridium satellite consultation is a large group of satellites providing data and and voice services to specialized satellite phones. The consultation exists of 66 active satellites in orbit. Iridium ase considered low-orbit satellites and are situated at hight of 781 km. The Iridium network is unique in the fact that it covers the whole earth, including oceans, airways and poles. The low orbit satellite differs from the balloons in the way that the satellites travel almost 40 times higher above earth \cite{iridium}. To be able to utilize the satellite one would need both a specialized satellite phone as well as a satellite dish in order to receive signal. With the balloons there are only need for an antenna to receive the signal. But then again the balloons does not offer a voice service. The balloons are intended as a cheaper, easier and simpler solution to the satellites.                                         

\section{Apple's Mesh Network}
In March 2014 a new iOS app was released, FireChat. This app enables the possibility to chat with people "off-the-grid" \cite{fireChat}. FireChat utilizes Apple's Multipeer Connectivity Framework introduced in iOS7 \cite{appleMesh}. Applications that communicates through this framework creates a mesh network similar to the one created by the Mesh Potatoes. 

The Multipeer Connectivity framework provides support for discovering services provided by nearby iOS devices using peer-to-peer Wi-Fi, infrastructure Wi-Fi networks and bluetooth to communicate with those services. This communication could either be message-based data, streaming data and resources such as files \cite{multipeer}. These technologies have a short range, but this range can be greatly extended by a chain of users that creates a mesh network, see section \ref{subsec:mesh} for more information about mesh networks. AirDrop is a product that have been out there for a little while and also utilizes mesh networks. The Main difference is that FireChat is fully decentralized and peer-to-peer. When there are multiple users in one area FireChat relay messages in the same way as Internet, from node to node, just in this case it is from phone to phone.  This, not only, enables two users to chat with each other without Internet connection, but also far beyond Wi-Fi and Bluetooth range from each other, using the chain of users (phones). For example if Bob is connected to Alice, and Alice is connected to Carol, Bob and Carol can send messages to each other. This chain can be indefinitely long. As long as no one device goes out of Wi-Fi range, all the devices can communicate with each other. 

This new framework will mainstream wireless mesh networking. Internet can now become available in places where it earlier was non existent. This could for example be a hotel basement, cave or rural areas where there are no cellphone towers, or disaster situations where Wi-Fi or mobile broadband  is no longer available. There are many benefits with the use of mesh networks. Mesh networks does not require a centralized infrastructure, there are no need for all the devices to be connected to a access point (as a router). Another benefit is that the mesh network is really easy to set up - everybody just uses the app FireChat (or similar applications like AirDrop), the network is created and everybody is connected. Simple as that! 

The possibilities for this feature are enormous. Both in the creation of applications but also the area of usage. In a lot of countries Internet and mobile broadband connection is extremely expensive, people might afford a used cellphone but not the cost to connect. With this new feature Internet connectivity can be spread through the mesh network needing only one node (phone) to have internet access. This way of spreading connectivity can open the possibility for people in rural, poorer areas like slums and small villages to stay connected. But not only the poor and rural areas can benefit from this new mesh-networking feature. Young people that does not have a phone can use an iPad or similar to talk to their friends. Or teens with restricted cell contracts can still connect with their friends, just with the help of the neighbours kids phone. Since FireChat enables communication without the use of internet it can be a useful tool to communicate privately and also to send sensitive data.
 
It is not only Apple that is seeing the enormous potential in main-streaming mesh networking. Google has also expressed that they are working on a home mesh network \cite{googleMesh}. FireChat and AirDrop is just the beginning. We believe more extensive and mind blowing applications are to come. 
 
\section{Different Scenarios Where a Quick Roll-out is Necessary}
%What are the specific need for the different scenarios? What adjustments are necessary?

Everyday there are situations all over the world that in some way affects the modern communications systems, or causes a need for one. These situations can range in everything from big natural disasters like the tsunami in Japan or the volcano outbreak in the Philippines, where either parts of the communication system is not functioning or there is a desperate need for one. To temporary refugee camps and IDP camps, and situations where a mobile tower is down, or blackouts. Also more festive situations can have use of the quick roll-out system. Imagine a big group gathered at a festival in another country. It is expensive to call or use mobile data, to be able to use cheap internet via the mesh Potatoes would save the users for a lot of money. 

\subsection{Natural Disasters}
%Asia og Sør-Amerika
%Philippines - contact Kenneth Bjerkelund to find out their needs for internet etc. makingchange.no. 
A Natural disaster is defined as; \textit{any event or force of nature that has catastrophic consequences, such as avalanche, earthquake, flood, forest fire, hurricane, lightning, tornado, tsunami, and volcanic eruption} \cite{naturalDisaster}.

All over the world there a relief organizations ready to go help out when a natural disaster would occur. These groups of people have the equipment, knowledge, experience and funding to help an area in desperate need. But how would that help it they do not know where they might be needed? Or if they hear a bout the disaster and the fist respond team are on place, how do they report back about the situation? How do they communicate with each other to work more effectively and help the ones in desperate need. There is no doubt that the need for a simple, fast and reliable communication system in necessary. 

When a natural disaster hits, one never knows the extent. Which makes it very difficult to predict how the communication systems would be affected. This unpredictability makes it important to always have a backup plan to the backup plan. 
%kanskje flere eksempler her?

History shows that cellphone service is not a reliable service during an emergency situation. During 9/11 the system became heavily overloaded, and when hurricane Katrina hit 70\% of the cell phone towers where knocked down. One might think that if they live in a big metropolitan that they would be safer \cite{disasterComm}. These are situations in the western world. 

When looking at the eastern, and often less developed world, that are even more vulnerable to natural disasters, the situations are different. According to \cite{DevelopingWorld, 360} developing countries are in an larger extent affected by natural disasters than the developed world. The reason for this can be explained by the economic status of the country. Both in how the country is prepared for the disaster and in how fast they can rebuild and recuperate after a natural disaster. Developing countries often lack the infrastructure needed to quickly and efficiently provide aids to the ones affected. According to Baxter \cite{360} , a natural disaster could set back a developing country many years in development.  

\paragraph{Hurrican Sandy}
Hurricane Sandy hit big parts of the Caribbean as well as the southeast parts of the Unites states at the end of October 2012 \cite{WikiSandy}. As many as 25 percent of the citizens in the affected areas lost cell phone coverage, and even more lost electricity. Emergency communication is a challenge in natural disasters, and often leave the public with out a way to call the the emergency number, but also makes it difficult for first responders (as fire fighters, police, etc) to communicate.  Satellite is sometimes used but is an expensive solution and they have more fixed restrictions, plus the fact that the equipment needed, phones and dishes, are expensive. No single communication system is fault free, and there always have to a backup to the backup. Satellite was used but the phones are expensive and the lines can be oversaturated if others parties are trying to connect to the network at the same time. A small aperture terminal (VSAT) trailer was also used to act like a satellite ground station. Finding a good spot for the trailer can be tricky, it requires clear view to the sky and can not be placed too close to a large building. The Red Cross launched an emergency preparedness application for smartphones. The application had a peak in downloads right before the hurricane hit, but when the commercial wireless network failed, they had to back to the old way of spreading information. Distributing paper files, going from house to house to check up on peoples welfare, give information word-by-mouth and using bullhorn \cite{hurricaneSandy}.

\paragraph{Philippines}

*Skal fylles ut mer her

November 8 2013 the typhoon Haiyan, a powerful tropical cyclone, struck and devastated parts of southeast Asia, in particular the Philippines. Haiyan is the strongest hurricane in wind speed ever recorded. The hurricane have the highest number for casualties, killing at least 6,268 people in the country alone \cite{wikiHaiyan}. International humanitarians and the government at the Philippines was warned about the storm in advance, but nobody could anticipate its viciousness. Some of the first teams on the spot was the communication experts, in order to coordinate, and make sure information was spread as desired.    \cite{disasterResponse} 

Clusters are groups of UN and non-UN humanitarian organisations that specialise in emergency repons in areas such as water and sanitation, health, shelter, logistics, food security and agriculture. The cluster approach was applied for the first time after the 2005 earthquake in Pakistan \cite{disasterResponse}.


\paragraph{Utilizing the emergency box.}
*detaljert forklaring på hvordan nødboksen kan settes opp for bruk i denne type scenario. 

\subsection{Temporary Refugee and IDP camps}

*Skal fylles ut mer her

Not all refugee and IDP camps are as well established, like the ones in Dadaab. Many camps are short-term, and are therefore in more need of a temporary communication system. In this case, setting up Mesh Potatoes in the camp to provide the refugees/IDPs with Internet access is an option. 

\paragraph{Utilizing the emergency box.}
*detaljert forklaring på hvordan nødboksen kan settes opp for bruk i denne type scenario. 

\subsection{Festivals}
Image you are at a music festival with your friends in a foreign country. There are thousands of people, and much activity. In a scenario like this there are many reasons why an Internet connection would be beneficial. You could loose your friends, have to inform your friends about something urgent, inform the staff if an emergency situation occur and so on. Since it is very expensive to send text messages/make phone calls or use mobile networks when you are abroad, it could be an idea to use Mesh Potatoes to provide the people at the festival with Internet access. This could be set up by the organizers in advance. Although this adds an extra cost to the organizers, the people attending the festival can save a lot of money by refraining from using the expensive services available on their cell phones. The organizers could add an extra fee to prize of the festival pass, and it would probably still be beneficial for the people at the festival. 

\paragraph{Utilizing the emergency box.}
*detaljert forklaring på hvordan nødboksen kan settes opp for bruk i denne type scenario. 

\subsection{Breakdown of Mobile Towers}

The 10th of June 2011 one of Telenor had problems with one of its servers in Oslo. This problem caused a down time of 18 hours and affected 3 000 000 Telenor users \cite{listeNedetid}. Not only was this the biggest problem Telenor have had since they opened their mobile network in 1993, but also the longest downtime and highest number of affected users recorded in Norway. In addition to this it all happened in a period with severe flooding in big parts of the eastern Norway, and made it difficult to reach emergency numbers. The fact that the problems occurred during the flooding just made the situation much worse \cite{TelenorNede}.

\paragraph{Utilizing the emergency box.}
*detaljert forklaring på hvordan nødboksen kan settes opp for bruk i denne type scenario. 


\subsection{Differences/relation between different scenarios}
*Sammenligning mellom scenarioer og oppsett. 
*Enkel "overføring" mellom scenarioer. Endringer som trengs. 


%What kind of adjustments are needed between the different scenarios?


The easiest way to get Internet to the \gls{mp} is by connecting it to the with an Ethernet cable to the jack port in the wall. This way of receiving internet requires no specific configuration. 



