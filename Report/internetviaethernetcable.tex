\subsubsection{Quickly Connect a Mesh Potato to the Internet via Ethernet Cable}


In order to perform this set up, a PC with Linux Ubuntu and wireless Internet and a Mesh Potato 2.0 is required. The last octet of the IP address for the MP is the unique number for each MP. In the following example we use 21, but this number must be different for each MP in a network. 

\begin{enumerate}
\item Connect the MP to the PC running Linux Ubuntu with a Ethernet cable. The Ethernet cable must be put into the LAN-port on the MP. 
\item Open Linux terminal and install telnet by entering the following commands: 
\noindent
\begin{lstlisting}[language=bash]
  $ sudo apt-get install telnetd
  $ sudo /etc/init.d/openbsd-inetd restart 
\end{lstlisting}
\item The default IP address of the MPs are 10.130.1.20, so in order to access the MP, the PC must be on the same subnet. To do this write in the terminal: 
\noindent
\begin{lstlisting}[language=bash]
  $ ifconfig eth0 up 10.130.1.120 netmask 255.255.255.0
\end{lstlisting}
\item Open a browser on your PC and write in "10.130.1.20". The SECN Web Interface should now appear. This assures you that you have contact with the MP.
\item In the interface, change the IP address of the MP to: 192.168.1.21. Press save, and then reboot in the interface. Wait for the system to reboot. 
\item Now that the IP address of the MP is changed from a local IP address, you also have to change the IP address of the ethernet port on your PC. Do this by writing this in the terminal:
\noindent
\begin{lstlisting}[language=bash]
  $ ifconfig eth0 up 192.168.1.2
\end{lstlisting}
\item Now that the PC and the MP are on the same subnet, you can telnet onto the MP by writing: 
\noindent
\begin{lstlisting}[language=bash]
  $ telnet 192.168.1.21
\end{lstlisting}
\item Open root shell in terminal by writing: 
\noindent
\begin{lstlisting}[language=bash]
  $ sudo su
\end{lstlisting}
You will get prompt for password. 
\item When in root shell, write: 
\noindent
\begin{lstlisting}[language=bash]
  $ iptables --table nat --append POSTROUTING --out-interface
   eth1 -j MASQUERADE
  $ iptables --append FORWARD --in-interface eth0 -j ACCEPT
  $ echo 1 > /proc/sys/net/ipv4/ip_forward
\end{lstlisting}
\item Open a editor on you PC (for example gedit). Paste the following text into a new file: 
\begin{framed}
\begin{verbatim}
#!/bin/bash
ARGS=2
if [ "$(id -u)" != "0" ]; then
    echo ""
    echo "    Sorry, this script must be run as root."
    echo "    Try sudo $0 name_of_wireless_interface
     name_of_ethernet_interface"
    echo ""
    exit
fi
if [ $# -ne "$ARGS" ]; then 
    echo ""
    echo "    This script sets up IP masquerading from your 
    ethernet port to your wifi Internet connection."
    echo "    usage: $0 name_of_wireless_interface
     name_of_ethernet_interface"
    echo "    e.g. $0 wlan0 eth0"
    echo ""
    exit
fi
wifi=$1
eth=$2

/sbin/iptables --table nat --append POSTROUTING
 --out-interface $wifi -j MASQUERADE
/sbin/iptables --append FORWARD --in-interface $eth -j ACCEPT
/bin/echo 1 > /proc/sys/net/ipv4/ip_forward
\end{verbatim}
\end{framed}

Save this file as "mp_masquerade.sh" in the directory /usr/local/bin. Make the file executable by entering the following in the terminal:
\noindent
\begin{lstlisting}[language=bash]
  $ sudo chmod +x /usr/local/bin/mp_masquerade.sh
\end{lstlisting}
\item Telnet into the MP and write:
\noindent
\begin{lstlisting}[language=bash]
  $ telnet 192.168.1.21
  $ route 
  $ route del default
  $ route add default gateway 192.168.1.2
\end{lstlisting}
\item Now you should be able to ping IPs on the Internet from the MP. Test this by writing "ping <IP address of a Internet page>". 
\item Next step is to get DNS working on the MP. In order to do this you must locate the file /etc/resolv.conf on your PC, and open in. You Copy the IP address stated in the nameserver line. Then open the SECN Web Interface, and locate the DNS field under Network under the tab Advanced. Paste the IP address into this field. Save and reboot from the interface. 
\item After all this, take out the Ethernet cable from the PC and put it in the wall. WALLA, you have Wi-Fi. Go nuts!
\end{enumerate}