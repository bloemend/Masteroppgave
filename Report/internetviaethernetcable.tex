
The easiest way to get Internet to the \gls{mp} is by connecting it to the with an Ethernet cable to the jack port in the wall. This way of receiving internet requires no specific configuration. 

The Mesh Potato is delivered preconfigured with IP address and name of the network (SSID) stated in the \gls{quick} box.

\begin{enumerate}
\item Make sure to have wireless contact with the \gls{mp}.

\item preform the following steps in order to wireless connect the PC to the \gls{mp}.
\begin{enumerate}
\item Press the Wi-Fi symbol in the top right corner on your Linux Ubuntu home screen, and press "Edit Connections".
\item  Under the tab "Wireless" choose the network called "vt-mesh" and press "edit"
\item In the field BSSID enter "02:CA:FF:EE:BA:BE".
\item Press "save" and close the window. 
\end{enumerate}

 

\item Connect the MP to wired Internet with an Esthernet cable. Using the "LAN" port on the \gls{mp} and the wall jack. 

\item Telnet into the \gls{mp}
\noindent
\begin{lstlisting}[language=bash]
  $ sudo su
  $ telnet 10.10.1.20
\end{lstlisting}

\item inside the telnet execute the following command
\noindent
\begin{lstlisting}[language=bash]
  $ udhcpc -i br-lan
\end{lstlisting}















, running Linux Ubuntu, with an Ethernet cable. The Ethernet cable must be put into the LAN-port on the MP. 

\item Enter the following command in the terminal:
\noindent
\begin{lstlisting}[language=bash]
  $ sudo su
  $ ifconfig eth0 up 172.31.255.253 netmask 255.255.255.252
\end{lstlisting}

\item Telnet into the \gls{mp}
\noindent
\begin{lstlisting}[language=bash]
  $ telnet 172.31.255.254
\end{lstlisting}

\item inside the telnet execute the following command
\noindent
\begin{lstlisting}[language=bash]
  $ udhcpc -i br-lan
\end{lstlisting}

\item 


\end{lstlisting}
\end{enumerate}
