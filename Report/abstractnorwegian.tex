\renewcommand{\abstractname}{Sammendrag}
\begin{abstract}
Village Telco er en organisasjon som har som mål å tilby en billig “plug-and-play”-løsning for data- og taletjenester på steder hvor ingen andre har muligheten, eller er villig til å tilby det. Denne løsningen blir levert ved å benytte et rimelig “fixed mesh WiFi” leveringssystem kalt Mesh Potato. Village Telco sin løsning kan bli anvendt hvor som helst i verden der mennesker ønsker å ta kontroll over sin egen kommunikasjonsinfrastruktur. Mesh Potato nettverk kan bli distribuert enten som en frittstående løsning, eller benyttes som en utvidelse i en allerede eksisterende teknologi. Village Telco sine eksisterende distribusjoner inneholder alt fra ti, til flere hundre Mesh Potatoes. 

Vi har rettet vårt arbeid mot bruken av Mesh Potatoes i mobile situasjoner. Vi har sett på forskjellige scenarier, alt fra krisesituasjoner og naturkatastrofer, til musikkfestivaler og midlertidige flyktningleire. Keith Williamson, en av de som jobber frivillig for Village Telco, har tidligere laget en boks omtalt som “go box”. I denne boksen benyttes den første generasjonen av Mesh Potato. Vårt ønske var å videreutvikle denne løsningen, og lage en mobil, og frittstående, løsning som kan bli brukt i alle de tidligere nevnte scenarier. Prototypen vi har laget er noe annerledes fra Williamson sin “go box”. Vi har benyttet den andre generasjonen av Mesh Potato, og vår boks inneholder alt som er nødvendig for at den skal kunne rulles ut raskt; solcellepanel, batteri, nødvendig kabler og brukervennlige manualer for konfigurering og oppsett. 

Vi har fastslått at mange av brukermanualene/beskrivelsene som finnes på Village Telco sin wiki er lite forklarende og vanskelig å forstå. Dermed ble en stor del av vår opppgave å forenkle disse beskrivelsene. Vi har laget manualer som beskriver hvordan man kobler en Mesh Potato til forskjellige aksessnettverk for å få Internettilgang. Fire testpersoner, både med og uten god teknisk forståelse, har testet manualene. Testprosessen ga oss verdifulle tilbakemeldinger, og dette førte til forbedringer i manualene. Vi har sett på prosessen rundt utrulling av vår løsning, og kommet med forslag til hvordan denne prosessen kan utføres raskest mulig. Vi endte derfor med å kalle løsningen vår for “QUICK” boks. 

Vi tror og mener at vårt arbeid og forskning har bidratt positivt til Village Telco-samfunnet. Vår prototype kan videreutvikles og introduseres for hjelpeorganisasjoner, og for andre mennesker som har en interesse av denne mobile kommunikasjonsløsningen. 


\end{abstract}