This is a description of how you set up and get started with the \gls{quick} box. 

The box contains the following items. Make sure that the box contains all these items before using it. 
\begin{itemize}
\item A Mesh Potato
\item A battery
\item A solar panel
\item A charging regulator
\item Various cables
\item A CD with Linux Ubuntu operating system
\item An USB with scripts to get Internet 
\item Descriptions on how to connect the Mesh Potato to different up-links in order to get Internet. 
\end{itemize}

The \gls{quick} box is delivered with a pre-configured Mesh Potato, and a fully charged battery. The battery can be charged by placing the solar panel in sunlight. 

\begin{description}
\item[] \textbf{Name of Mesh Potato (SSID):} MP2_21.
\item[] \textbf{IP-address of the Mesh Potato:} 192.168.1.21
\item[] \textbf{Password:} potato-potato 
\end{description}

The password is the default password used on Mesh Potatoes, and it can be changed in the user interface for a more secure option. The SSID can also be changed there if that is preferable. 

\begin{enumerate}
\item Connect the wires to the battery (the red one to plus and the black one to minus). The Mesh Potato should then automatically turn on. 
\item To provide Internet to the mesh network, follow the attached descriptions for the specific up-link type you have available. 
\end{enumerate} 

In order to preserve the lifetime of the battery, it can be smart to disconnect the wires from the battery while the box is not in use. A fully charged battery has a lifetime of approximately 83 hours. This is when the solar panel is not connected to the battery. Take into consideration that the charging time when the battery is completely discharged is approximately 10 hours. If the \gls{mp2} is connected while charging, 1 hour is added to the charging time. 
