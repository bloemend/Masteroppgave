This is a description of how you set up and get started with the emergency box. The box contains a Mesh Potato, a battery, a solar panel, a charging regulator, various cables, a CD with Linux Ubuntu operating system, an USB with scripts to get Internet and descriptions on how to connect the Mesh Potato to different up-links in order to get Internet. 

The emergency box are delivered with a pre-configured Mesh Potato, and a fully charged battery. The battery can be charged by placing the solar panel in sunlight. 

\begin{description}
\item[] \textbf{Name of Mesh Potato in the box (SSID):} MP2_21.
\item[] \textbf{IP-address of the Mesh Potato:} 192.168.1.21
\item[] \textbf{Password:} potato-potato 
\end{description}

The password is the default password used on Mesh Potatoes, and it can be changed in the user interface for a more secure option. The SSID can also be changed there if that is preferable. 

\begin{enumerate}
\item Connect the wires to the battery (the red one to plus and the black one to minus). The Mesh Potato should then automatically turn on. 
\item To provide Internet to the mesh network, follow the attached descriptions for the specific up-link type you have available. 
\end{enumerate} 