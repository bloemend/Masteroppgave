If there is a satellite dish available, getting Internet is fairly easy. Satellite internet does not use telephone lines or cable systems, but instead utilizes a  satellite dish to get two-way data communication. Satellite Internet consists of a satellite dish, a modem (both for uplink and downlink), a coaxial cable between the dish and the modem, and an Ethernet cable from the modem to the laptop. 

Make sure that the cables are connected correctly. The coaxial cable must be connected from the satellite to the modem. There are two alternatives for where to plug in the Ethernet cable. The one end must be connected to the modem, and the other can either be connected to a \gls{mp} or a PC.

If you choose to connect the Ethernet cable directly to the \gls{mp}, the set-up in section \ref{subsec:cabledInternet} must be executed. 

Another option is to connect the Ethernet cable from the satellite modem to a PC. One drawback with this approach is that the PC must have two Ethernet ports (one to connect to the MP, and one to the modem). If two Ethernet ports are available, the set-up for getting Internet to the \gls{mp} is identical to the set-up for "How Connect the MP02 to Internet via PC Getting Wireless Internet from Landline or Cellular" (section \ref{subsec:internetviaPC}). Although this is a set-up from a PC getting wireless Internet, it also works with a PC getting cabled Internet. Just make sure that the right interfaces (name of port) is chosen in step 3 and 5. To get an overview over available interfaces, the following command can be executed in the terminal window: "iwconfig". 

%Locate the coaxial cable that comes from the dish. After the modem is installed, you plug the coaxial cable to "SATELLITE IN" and "SATELLITE OUT" ports on the modem. Then plug the Ethernet cable to the modem and to your computer. 

%So, basically after this is set-up, you can follow the description of how to get Internet to the MP via a PC to get Internet from satellite to the mesh network. 