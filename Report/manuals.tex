\chapter{Roll-out of the QUICK Box}
\label{chp:manuals} 

As mentioned, one area of focus is the process of quick roll-out. There are many aspects that can be included in order to speed up the roll-out process and make it as easy as possible. We will now present some of our ideas to meet this requirement.


\section{Scripting}
A script is a list of commands that can be executed without user interaction, in other words, to automate a process. In order to connect the mesh network to Internet a list of commands have to be executed. One idea to speed up the process of setting up the network was to create a script to automate this process. One way to do this could be by creating a self-executable script that could be included in the \gls{quick} box on for example on an USB-stick. There would have to be a different script, and different USBs, for each up-link type. 

We made a script that simplifies the process of getting Internet to a Mesh Potato via a PC (see section \ref{subsec:internetviaPC} for step-by-step description). This script can be found in Appendix \ref{chp:appendixD}. The PC was running Linux Ubuntu and had a wireless Internet connection. Some easy steps must be conducted to run the script:
\begin{enumerate}
\item Make sure the \gls{mp} is connected to the PC via Ethernet cable.
\item In the terminal go to the directory where the script is located and enter the following command:
\noindent
\begin{lstlisting}[language=bash]
  $ sudo su
  $ sh scriptviaPC.sh
\end{lstlisting}
\end{enumerate}

\section{Distributing Numbers}
The \gls{mp2} Basic does not have the option to connect to a phone. Hence the issue with telephone number distribution is irrelevant. Later in 2014 Village Telco will release a new version of the \gls{mp2}, \gls{mp2}-Phone. MP2-Phone will be identical to \gls{mp2} Basic, just with an \gls{fxs} daughterboard. With this new version the issue of number distribution appears. 

The \gls{quick} box could be delivered with several \glspl{mp}, where each \gls{mp} is marked with the pre-configured IP address. Since it is not possible to connect a phone to the \gls{mp} there is no issue of number distribution. The box could for example contain 5 \glspl{mp}, where one would be connected to an up-link, while the other ones would be strategically placed in order to spread the Internet access further. 

When a Village Telco is set up today, telephone numbers are distributed by updating a spreadsheet with name and number to the users. These spreadsheets are printed out and delivered to everyone. This is a system that might seem cumbersome, but serves its purpose. If new nodes are added to the network or any changes are made, new sheets have to be printed out and delivered to everyone. This way of spreading telephone numbers might be more difficult with the go-box. 

One option is to continue with the number distribution approach in use today. The suitcase could contain 5 \glspl{mp}, all \glspl{mp} are marked with its unique \gls{ip} address. There will be attached a list with the \gls{ip} addresses of the other Mesh Potatoes in the suitcase. When setting up the network the names can easily be filled in on each Mesh Potato. This will then be the telephone list.  

Another approach would be to integrate the distribution of phone number in the web interface. A new feature could be implemented in the interface. This feature would discover the other \glspl{mp} in range, also in range through other \glspl{mp}. All \glspl{mp} would be displayed with name of the \gls{ssid}, \gls{ip} address, where the last octet is the telephone number and the name of residence or user. This name could be edited by the master user or by the user themselves. Each \gls{mp} in the suitcase are pre-configured and set up, they are also set up with security and a password to enter the web interface. So in order for a user to enter the web interface they have to enter the password to get access. Once inside they can see other \glspl{mp} in range and also put in their name for the other \glspl{mp} to see. The password and security is set up so that no other than the main user of the \gls{mp} can change the name. 

\section{Manuals}
The following section contains manuals describing how to get started with the \gls{quick} box, and how to connect it to different up-links in order to provide Internet to the network. All the manuals below will be laminated, and provided in the \gls{quick} box. 

\subsection{Get Started - How to Use the Box}

This is a description of how you set up and get started with the \gls{quick} box. 

Make sure that the \gls{quick} box contains all these items: 
\begin{itemize}
\item A Mesh Potato
\item A battery
\item A solar panel
\item A charging regulator
\item Ethernet cable
\item A CD with Linux Ubuntu operating system
\item A USB-stick containing a script called "scriptviaPC.sh"
\item Descriptions on how to connect the Mesh Potato to different up-links in order to get Internet. 
\end{itemize}

The \gls{quick} box is delivered with a pre-configured Mesh Potato, and a fully charged battery. The battery can be charged by placing the solar panel in sunlight. 

\begin{description}
\item[] \textbf{Name of Mesh Potato (SSID):} MP2_21.
\item[] \textbf{IP-address of the Mesh Potato:} 192.168.1.21
\item[] \textbf{Password:} potato-potato 
\end{description}

The password is the default password used on Mesh Potatoes, and it can be changed in the user interface for a more secure option. The SSID can also be changed there if that is preferable. 

To set up the \gls{quick} box: 
\begin{enumerate}
\item Connect the wires to the battery (the red one to plus and the black one to minus). The Mesh Potato should then automatically turn on. 
\item To provide Internet to the mesh network, follow the attached descriptions for the specific up-link type you have available. 
\end{enumerate} 

In order to preserve the lifetime of the battery, it can be smart to disconnect the wires from the battery while the box is not in use. A fully charged battery has a lifetime of approximately 83 hours. This is when the solar panel is not connected to the battery. Take into consideration that the charging time when the battery is completely discharged is approximately 10 hours. If the \gls{mp2} is connected while charging, 1 hour is added to the charging time. 


\subsection{How Connect the MP02 Directly to Cabled Internet}
\label{subsec:cabledInternet}

The easiest way to get Internet to the \gls{mp} is by connecting it to the with an Ethernet cable to the jack port in the wall. This way of receiving internet requires no specific configuration. 




\subsection{How Connect the MP02 to Internet via PC Getting Wireless Internet from Landline or Cellular Network}
\label{subsec:internetviaPC}
If you have a PC supporting wireless Internet, there are different ways of getting wireless Internet to it. You can get WiFi on your PC from a router with landline connection, or the PC can establish a wireless connection to a access point for example in form of a cell phone. The cell phone may have a cellular network available. On most new smart phones, you can set your phone to act as an access point (AP), so that other devices can connect to it and get Internet. You can off course connect directly to this AP, but then the MP does not get Internet, and can not spread it further on to several neighbour MPs. The following set-up works for both types; either if you connect to a regular wireless router that gets Internet from for instance xDSL or if you connect to a AP that have a cellular network (3G, 4G) available. 

In order to perform this set up, a PC with Linux Ubuntu with wireless Internet, and a Mesh Potato 2.0 is required. The last octet of the IP address of the MP, is the unique number for each MP. The Mesh Potato is pre-configured with an unique IP address which is stated on the MP. In the following example we use "x" as the last octet. When conducting this description please change the x with the last octet written on your MP.

\begin{enumerate}
\item Connect the MP to the PC, running Linux Ubuntu, with an Ethernet cable. The Ethernet cable must be put into the LAN-port on the MP. 

\item Open Linux terminal and install telnet, dns and iptables by entering the following commands: 
\noindent
\begin{lstlisting}[language=bash]
 $ sudo su
 $ apt-get install telnetd
 $ /etc/init.d/openbsd-inetd restart 
 $ apt-get install dnsmasq
\end{lstlisting}

\item The Mesh Potato will be pre-configured and the IP address 192.168.1.x. In in order to access the MP, the PC must be on the same subnet. To do this write in the terminal: 
\noindent
\begin{lstlisting}[language=bash]
  $ ifconfig eth0 up 192.168.1.2
\end{lstlisting}

\item Open a browser on your PC and type in "192.168.1.x" in the URL field. The SECN Web Interface should now appear. This assures you that you have contact with the Mesh Potato. Changes in the interface will be described further down, so do not close this window.  

\item Go back to the terminal and write the following commands in order to set up the ip tables correctly. You might have to change the "eth0" and "eth1", depending on how your laptop is set up. The eth0 in the following commands is equivalent to the interface of the Ethernet port connected to the MP, while the eth1 is the interface to the wireless network. 
\noindent
\begin{lstlisting}[language=bash]
  $ iptables --table nat --append POSTROUTING --out-interface
   eth1 -j MASQUERADE
  $ iptables --append FORWARD --in-interface eth0 -j ACCEPT
  $ echo 1 > /proc/sys/net/ipv4/ip_forward
\end{lstlisting} 
\begin{itemize}
\item If you mess up in this step, accidentally write something wrong etc., the following commands will reset the ip tables, and you may try step 5 again.
\noindent
\begin{lstlisting}[language=bash]
  $ iptables --table nat --flush
  $ iptables --flush
  $ iptables --delete-chain
\end{lstlisting}
\end{itemize}  

\item Telnet into the \gls{mp} and configure the default gateway by entering the following commands
\begin{lstlisting}[language=bash]
  $ telnet 192.168.1.x
  $ route 
  $ route del default 
  $ route add default gateway 192.168.1.2
\end{lstlisting} 

\item Go back to the web interface and click on the "Advanced"-tab at the top of the page. Change the following parameters under "DHCP Server":
\begin{itemize}
\item Tick the box "Enable DHCP Server".
\item Remove the tick from "Use device IP".
\item Change the address in "Gateway Router" to "192.168.1.2".
\item Press "Save" at the bottom of the page. 
\end{itemize}

\item Internet should now be available in the mesh network. A device can connect to the network with the SSID (name of network) stated on the emergency box. This SSID is also stated in the web interface under "WiFi Access Point".  
\end{enumerate}

 
\input{internetviacellular}

\subsection{How Connect the MP02 to Satellite}
If you have a satellite dish available, getting Internet to your PC from the dish is not difficult. In addition to the satellite dish, you need a modem, coaxial cable, Ethernet cable and software. Locate the coaxial cable that comes from the dish. After the modem is installed, you plug the coaxial cable to "SATELLITE IN" and "SATELLITE OUT" ports on the modem. Then plug the Ethernet cable to the modem and to your computer. 

So, basically after this is set-up, you can follow the description of how to get Internet to the MP via a PC to get Internet from satellite to the mesh network. 
