\chapter{Interview with CARE}
\label{chp:appendixA} 

This appendix contains the summary from the interview conducted with Mary Muia, who works for CARE International in Kenya. She works as a program assistant for the Refugee Assistance Programme in Dadaab.


\begin{enumerate}

\item Approximately how many people are there in the Dadaab refugee camp? And how long have it been in operation?

The Dadaab complex of refugee camps, considered the world’s largest, was created in 1991 by the Government of Kenya and UNHCR to host Somali refugees displaced by civil war. Over the years, the camps have also hosted other nationalities from the Horn of Africa, the Great Lakes and East Africa regions but they constitute less than two percent of the camp population. The original camps were Dagahaley, Ifo and Hagadera and were intended to host 90,000 refugees. However, in 2011, there was an influx of new refugees from Somalia due to severe drought and new camps were created; Ifo 2 and Kambioos, to cater to the over 175,000 new arrivals and at the peak of the influx in 2011, the camps hosted more than 463,000 refugees, including some 10,000 third-generation refugees born in Dadaab to refugee parents who were also born there. However, in 2013, UNHCR and its partners conducted a verification exercise to ascertain the current population since some of those who had arrived in 2011 due to the famine had returned home. As at February, 2014, the current population stands at 369,294.

\item How do you connect and communicate with the outside world?

CARE as an organization has invested in communication systems in liaison with Internet Service Providers in the capital city of Nairobi who ensure that all staff have access to internet for both official and social usage.

\item How are the communication inside the camp (communication flow)?

Several telecommunication firms in Kenya have put up their machinery in the area thus there is access to both mobile communication and access to internet services. There are also radio station services and access to digital televisions. CARE uses telephone services to reach out to refugee staff (50\%) of the refugees have access to mobile phone services - either owned or through a bureau) posters and radio to reach out to its beneficiary population. In addition, there is word of mouth done through loud speakers during major gatherings like food distribution days and also road shows within the camps.


\item How does the refugees receive information?

As 3 above.

\item Can you exlain what happens when a new person enters the camp?

Upon arrival, a new refugee would report to a UNHCR reception desk whereby they are given temporary registration pending full registration and location of their relatives is they have any already in the camp. UNHCR fully briefs the new arrival on all the services available and which Agency is handling what service. Immunizations, medical attention, emergency food supply, tarpaulins, sleeping mats, jerrycans for fetching water and kitchen sets are issued to such new arrivals to help them start their new lives in the camps. UNHCR then hands over the new arrivals to the respective Agency doing camp management in the specific camp they are allocated so that they can be shown where to pitch their tents. The camps are well demarcated into numbered sections and blocks thus at any given time, UNHCR would inform you where a particular refugee resides and the family size. Each Agency working in Dadaab has their own mode of communicating the services they provide to their target beneficiaries. However, UNHCR holds regular meetings with the refugee leaders of each respective camp whereby information is shared with them for dissemination to the entire refugee population.


\item What are the biggest challenges in a refugee camp?

Lack of enough space to accommodate everyone and lack of enough funds to take care of all the needs of the refugees.


\item What is the biggest challenge when it comes to communication/information spreading in the refugee camp?

Language barrier between the humanitarian staff and the refugees since many of the staff do not speak/understand the Somali language while 95.6\% of the refugees are Somali. Internet and telephone service outages are also common in the area and response by the service providers sometime take a while.


\item What means of communication do you use in the refugee camp?

Mobile phones and computers for both telephone and internet access. Radios and television services.

\item We have the impression that there are not many telecom providers offering telecommunication services in Africa, and hence little competition. Which in general makes the prices higher. How does the ones living in the camp afford to have a phone? 

(a) There are two main telecom provides here so yes, little competing thus high rates (b) Many refugees who have been here for over a long period of time have established small scale business (some supported by the NGO’s i.e. IGA’s (Income Generating Activities) thus make some little profit.  Others have established business through the support of their relatives who have been resettled in other countries thus send them some cash while others who may have been businessmen back in Somalia made it to take some of the cash they had at the time of fleeing their country.

\item  Is it "Internet cafes" that people have to pay to be able to use?

Yes some refugees have set up small internet cafes in the markets thus people who need the services have to pay for it.  CARE like other NGO’s here has Community Development Projects which include ICT training where we train the youth on ICT and upon successful completion, we support then by providing them with start-up kits to establish their own small cafes for both business and training others youth.

\item How long does it take to set up a communication system?

N/A since I am not a technical person


\item Do you use video surveillance?

No

\item Have you heard of something called Freedom fone?

No

\item Have you heard of the company Village Telco?

No


\item Do you have anything else to add that can be of interest for our master thesis?

No

\end{enumerate}