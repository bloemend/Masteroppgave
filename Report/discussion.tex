\chapter{Discussion}
\label{chp:discussion} 

Essentially the discussion chapter tells your reader what your findings might mean, how valuable they are and why. I remember struggling with this section myself and, looking back, I believe there were two sources of anxiety

How has the field's knowledge been changed?

Are the obtained results as expected? Why/why not?
Comparison with similar works. 

Nytt:
-Gjort det med MP2
-Laget forklaringer på hvordan få internett med mp2, det finnes ikke fra før såvidt vi veit. 
-laget noen script, brakt nye tanker inn om hvordan ting kan gå raskt. 
-samlet informasjon og gjort forklaringene mer brukervennlig.

- Hva gjør vi og hvorfor?

Up until today the Mesh Potato has mainly been used to create a permanent communication system in villages all over the world. Especially in rural areas where the existing communication systems are expensive and the coverage is not satisfying. There is no doubt that the mesh networking is an up and coming mean of communication. This fact is  by Apple implementing the Multipeer Connectivity framework in their newest OS, and proves that this is a way of staying connected is a feature they want to invest in. In a world that is becoming more and more technological, there are still places and people that are not connected, and there are still places where internet connection does not reach. 

Even though the Mesh Potato installations today are permanent, this is no limitation. So the question is how could the Mesh Potato be utilized as a mobile installation, and in what situations would there be a need for one? When a natural disaster occurs there are many examples expressing that communication is an issue. Just look at the situation during hurricane sandy, the hurricane in the Philippines and the tsunami in Japan. Mobile towers are down, internet connection are lost, and the power is unavailable. In some situations there exist big difficulties, people have to walk long distances in order to find an Internet café or even to receive cell phone coverage. Which makes the coordination process with relief organizations hard and time consuming. Natural disasters are not the only scenario where a mobile communication system could be of interest. We have also looked at the possibility to use this mobile installation on festivals and in temporary refugee camps. 

Village Telco provides people with an extensive and comprehensive Wiki page. Unfortunately this page is not very well structured and can be apprehended as confusing, as well as little user friendly. In addition to this, the majority of the descriptions provided on the Wiki are directed towards the Mesh Potato v1.0. We started our research by setting up a network existing of the first generation Mesh Potato, and then moved on to the second generation. The second generation are improved on many areas, making the MP2.0 both faster and easier to use. Hence the descriptions are outdated and too complicated. A lot of the descriptions for MP1.0 requires the use of shell commands, while with MP2.0 more configurations can be done using the web interface. Shell commands are not well known to "average Joe", and leaves rooms for many mistakes, and a lot of confusion. It may be hard to undo actions conducted using shell commands, especially if you do not have any knowledge about it. We have simplified these descriptions, both in order to make them more understandable for the user and to direct them toward the second generation of the Mesh Potato. When a Village Telco is set up there are usually some with technological understanding that is in charge, but with the \gls{quick} box we want to make it possible for anyone to use, and that the descriptions are as easy and explanatory as possible. 


One of Village Telco's volunteers, Keith Williamson have created and tested the use of a go-box in disaster relief scenarios. This box was created with the first generation of the Mesh Potato, and tested during a small exercise in Maine, USA. The main difference between the go-box and the \gls{quick} box that we have created is the fact that we have created our box based on the second generation. At time of writing only the basic version of the MP2 is available. This basic version does not contain the possibility to connect a phone, this feature will become available in the next version of the second generation, available for sale summer 2014. Another difference is the that we are including everything needed for an installation. This includes a solar panel, necessary cables, and full descriptions on how to use the box and hos to connect to different up-links. There is no use having all the descriptions on the internet when you do not have access to internet. The \gls{quick} box can be used both to create an internal network in an area. This network will not be connected to the outside world but will make it possible for people to talk and send data between each other. The other option is to connect the \gls{quick} box to an uplink, this could be wired, wireless or satellite Internet. We believe that simplification and making information more accessible is a huge step in the direction of receiving more more users. 

One aspect that is of high importance when talking about a mobile way of creating a mesh network is the aspect of quick roll-out. By using the scripts we created, and maybe in the future created self-running scripts, the process will be automated to a higher extent and easier for the user to use. Especially the users that are less technical. 

Unfortunately we were not able to travel to one of the Village Telcos that are in operation. This implies that our studies are theoretical and based on previous work and the knowledge from these. As well as the studies we have done on the second generation \gls{mp}, in forms of easy and descriptive explanations on how to use. We believe that our study can lay the basis for future work in making the \gls{mp} more suitable for use in mobile situations, and further develop the \gls{quick} box. 

We tested these descriptions on different people of different age and technological knowledge. Testing always retrieves valuable information and shed light on issues we had not previously taken into account. 

*Skrive om testingen og hva folk syntes, fikk de det til? var det vanskelig? hva var det som var vanskelig? bla bla bla





 



