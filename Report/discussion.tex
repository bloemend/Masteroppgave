\chapter{Discussion}
\label{chp:discussion} 

Up until today the Mesh Potato has mainly been used to create permanent communication infrastructures in villages all over the world. The deployments have mostly been in rural areas where the existing communication systems are expensive and the coverage is not satisfactory. There is no doubt that mesh networking is an up-and-coming mean of communication. One example is Apple implementing the Multipeer Connectivity framework in their newest iOS. The fact that a large company, like Apple, invests in such a technology is a major driving force for the technology itself. During the last decade Apple has been a front-figure for innovation in the technological world. It is clear that other companies have tried to emulate Apple's technological contributes to the market, both in features and appearance. The iPad is a good example of this, since Apple were the first to introduce a tablet that caught customers' attention. The iPad became extremely popular, and within short time other big companies, like Samsung, came with similar products. This indicates that Apple is a trendsetter, and that by introducing the Multipeer Connectivity framework this will help mesh networking become more known and a more prominent mean of communication in the future. In a world that is becoming more and more technological for every day that passes, there are still places and people that are not connected, and there are still places without Internet connection. 

Even though the Mesh Potato deployments today are permanent, this is not a limitation. The question is how could the Mesh Potato be utilized as a mobile installation, and in what situations would there be a need for a communication system like this? When a natural disaster occurs, there are many examples expressing that communication becomes an issue. Just look at the situation during hurricane Sandy, the typhoon in the Philippines and the tsunami in Japan. Mobile towers were down, internet connection were lost, and there were power outages. People might have to walk long distances in order to find an Internet café or even to receive cell phone signal. The lack of a communication infrastructure makes the coordination process, both with, and for the relief organizations difficult and time consuming. Natural disaster are not the only scenario where a mobile communication system could be of interest. We have also looked at the possibility to use this mobile installation on festivals and in temporary refugee camps. 

Village Telco provides people with an extensive Wiki page. Unfortunately this page is not very well structured and can be apprehended as confusing, as well as little user friendly. In addition to this, the majority of the descriptions provided on the Wiki are directed towards the \gls{mp1}. We started our research by setting up a network existing of \glspl{mp1}, and then moved on to the second generation. The second generation are improved on many areas, making the \gls{mp2} both faster and easier to use. Hence the descriptions are outdated and too complicated. A lot of the descriptions for \gls{mp1} requires the use of shell commands, while with \gls{mp2} more configurations can be done using the \gls{secn} web interface. Shell commands are not well known to "average Joe", and leaves room for many mistakes, and a lot of confusion. It may be hard to undo actions conducted using shell commands, especially if you do not have any knowledge about it. We have simplified these descriptions, both in order to make them more understandable for the user and to direct them towards the second generation of the \gls{mp}. When a Village Telco is set up there are usually some with technological understanding that is in charge. With the \gls{quick} box we want to make it possible for anyone to set-up the network, since the descriptions are as easy and explanatory as possible. 

One of Village Telco's volunteers, Keith Williamson, has created and tested the use of a "go box" in disaster relief scenarios. This box was created with the first generation of the Mesh Potato, and tested during a small exercise in Maine, USA. The main difference between the "go box" and the \gls{quick} box is the fact that we have created our box based on the second generation. At time of writing only the basic version of the \gls{mp2} is available. With the basic version it is not possible to connect a phone, but this feature will become available in the next version of the second generation (available for sale summer 2014). Another difference is the that the \gls{quick} box includes everything needed for an installation;  a solar panel, necessary cables, and full descriptions on how to use the box and how to connect to different up-links. There is no use having all the descriptions available on the web, if you do not have access to Internet. The \gls{quick} box can be used both to create a local network without Internet access and a network with Internet access. The local network will not be connected to the outside world, but will make it possible for people to talk and send data between each other. The other option is to connect the \gls{quick} box to an up-link, either through landline, cellular network or satellite. We believe that simplification and making information more accessible is a huge step in the direction of gaining users. 

One aspect that is of high importance when talking about a mobile way of creating a mesh network is the aspect of quick roll-out. By using scripts, and maybe in the future self-running scripts, the process will be automated to a higher extent and will make it easier for users to wield. Automation will make it especially easier for the users that are less technical. 

Unfortunately, we were not able to travel to one of the Village Telcos that are in operation. This implies that the study we have conducted are theoretical and based on previous work and the experiences made from these. Our study is also based on the \gls{mp2}, in terms of easy and descriptive explanations on "how to use". We believe that our study can lay the basis for future work in making the \gls{mp} more suitable for use in mobile situations, and further development of the \gls{quick} box. 

We tested the different manuals for providing Internet to the \gls{mp} on four people of different age, gender and technological knowledge. Testing always retrieves valuable information and shed light on issues we had not previously taken into account. Based on the results presented in Chapter \ref{chp:test}, many interesting findings and observations were made. We will present the observations while looking at different groupings of the test persons; non-technical versus technical and men versus women. 

One thing we observed was that the non-technical test persons preferred manual 1, manual for connecting the MP02 directly to cabled Internet. This manual consists mostly of GUI-interaction, and little use of the terminal. We think this is the reason for why the non-technical test persons preferred this manual. Most every day interactions done with computers are preformed using the Graphical User Interface (GUI) provided by the different operating systems.  This is therefore a more well-known type of interaction than the terminal. We observed the non-technical test persons showed some hesitation when they realized that they had to use a different operating system, Linux, than they are used to. The matter did not get better when they where asked to perform terminal commands. This was the first time using the terminal for both of the non-technical persons. Using GUI in an unknown operating system might also be less frightening than using the terminal. 

The test persons with technical knowledge and background, on the other hand, chose manual 3 (script for getting Internet via PC) when they where asked with manual they preferred. They showed confidence when faced with the use of both Linux and the terminal. It was clear from our observation that this was not the first time the test persons with technical knowledge and background utilized Linux and terminal commands. An observations that backs up this assumption is the fact that they used keyboard shortcuts (Ctrl+Alt+t) to open the terminal window. We also observed that one of them used keyboard shortcuts (arrow up to get the latest command used) in the terminal, and one of them knew the commands in order to find the wireless interface and to toggle between directories (also inside the terminal). None of these actions were stated in the manuals.  

Another observation made while comparing the non-technical and the technical test persons is that the technical test persons became more caught up in the different commands and their output. They paid more attention to the content of the commands. An example of this is the command "route del default" in manual 2 (manual for connecting the MP02 to Internet via PC getting WiFi from landline or cellular network). This command prints out a short message that could be perceived as an error-message, although it is not. One of the technical persons stopped up after this message appeared, and thought something had gone wrong. The non-technical persons did not pay attention to this output. In fact, they did not pay attention to any outputs at all. When a user telnet into the \gls{mp}, a large welcome figure appears in the terminal window. The technical test persons were fascinated by this, and commented on it. The non-technical test persons, on the other hand, did not show any visible expression or fascination towards this. This could be because they simply did not notice it, or did not care. We found this strange, cause we thought that the graphical elements of the set-up would be of more fascination and more conspicuous. While performing manual 3 (script for getting Internet via PC) two pop-up windows appear during the set-up. All the test persons expressed some hesitation when seeing these pop-up windows. One of the non-technical test persons immediately closed the first window when it appeared. It seemed like she got confused when the pop-up windows appeared. It could be that she thought it was something not related to the script and wanted to close the window to not mess up what she was doing. The technical test persons were more interested in how the \gls{mp} and the set-ups work. During the set-up they asked questions and showed curiosity regarding the tests they where performing. This was not the case for the non-technical test persons. This might be because they have enough knowledge to ask relevant questions. Internet is a commonly used phenomenon, but very few have detailed knowledge about how it works. The observation mentioned (the fact that the non-technical does not pay attention to the output) is not necessarily a negative fact. Since they have little knowledge they do not question the commands or get caught-up in insignificant things. In other words, too much technical knowledge can result in unnecessary trouble and hesitation. 

When comparing the men with the women, we did some interesting observations. Men are typically more confident in own ability, while women often tends to be more cautious to make sure they do the right thing. Our observations verifies this. We immediately noticed that the men seemed more confident, and when provided with the different components immediately wanted to plug everything together, before even opening the manual. The women, on the other hand, thoroughly read the manual. Although the women more carefully read the manuals, the men where the ones paying most attention to proof-reading the commands in the terminal before executing them. This resulted in less misspelt commands for the men, hence less redos. 
